\section{Introduction}
When examining complex physics problems, partial differential equations often occur. Theese equations are often difficult and in some case impossible to calculate numerically.
Since it is desired to obtain a solution to such problems, we can investigate how a solution might look. This is done by looking at smaller parts of the domain where we can restrict the problem,
from theese restrictions we are able to approximate solutions, and by combining theese solutions we can obtain a approximated solution to the original problem, this process is called the Finite Element Method.
Furthermore we are able to give an error estimate of the approximated solution, which is important when we want to know how accurate our solution is.