\chapter{Introduction}
When examining complex physical problems, partial differential equations often occur. 
These equations are often difficult and in some case impossible to calculate algebraically.
We therefore approximate the solution numerically.
One way to do this is the Finite Element Method.
This is done by looking at smaller parts of the domain where we can restrict the problem,
and from these restrictions we are able to approximate solutions. 
By combining these solutions we can obtain an approximated solution to the original problem. 
This process is called the Finite Element Method.
Furthermore we are able to give an error estimate of the approximated solution, 
which is important when we want to know how accurate our solution is.
In this project we will examine the basic theory regarding the Finite Element Method and 
use it in a numerical analysis.