\begin{thmx}{Friedrichs' Inequality}
Let $\Omega$ be contained in a hypercube with sidelength $s$. Then 
\begin{equation*}
    \| v \|_0 \leq s |v|_1 \quad \quad \forall v \in H_0^1(\Omega).
\end{equation*}\label{thm:friedrich}
\vspace{-8mm}
\end{thmx}
\begin{bev}
We will start by assuming $v \in C_0^{\infty}(\Omega)$, and later use the 
completeness of $H_0^1(\Omega)$ to finish the proof.
Let $W = \{ (x_1, \ldots, x_n) \mid 0 < x_i < s \}$ be the box 
which contains $\Omega$, and let $v(x)=0$ \, $\forall x
\in W\setminus \Omega$. Otherwise we can simply translate $\Omega$ such that 
this is true.
 We can then write 
\begin{equation*}
v(x)=v(0, x_2, \ldots, x_n) + \int_{0}^{x_1} \partial _1 v(t,x_2, \ldots, x_n) dt.
\end{equation*}
Due to the definition of $W$, the first term (the boundary term) 
disappears. Applying the Cauchy-Schwartz inequality results in 
\begin{align*}
    |v(x)|^2 &= \left|\int_{0}^{x_1} \partial _1 v(t,x_2, \ldots, x_n) dt \right|^2 \\ 
    &= |{(1, \partial _1 v)}_{L^2([0,x_1])}|^2 \\
    & \leq \int_{0}^{x_1} 1^2 dt \int_{0}^{x_1}|\partial_1 v(t, x_2, \ldots, x_n)|^2 dt \\ 
    &\leq s \int_{0}^{s}|\partial_1 v(t, x_2, \ldots, x_n)|^2 dt. 
\end{align*}
After expanding the integral and getting the last inequality, we know 
the right hand side is independent of $x_1$, and so by integrating we find 
\begin{equation*}
    \int_{0}^{s}|v(x)|^2dx_1 \leq s^2\int_{0}^{s}|\partial _1 v(x)|^2dx_1.
\end{equation*}
This can be done for every variable, and finding the integral over the entire 
domain gives us 
\begin{equation*}
    \|v\|_0 = \int_{W}|v|^2 dx \leq s^2 \int_{W} |\partial _1v|^2 \leq s^2 |v|_1^2.
\end{equation*}
Since $C^\infty_0(\OO)$ is complete in $H_0^1(\OO)$, for every $u \in H_0^1(\OO)\setminus C^\infty_0(\OO)$
there exists some Cauchy sequence of smooth functions $u_k \in C^\infty_0(\OO)$
 such that 
\begin{equation}
    \lim_{k \to \infty} \| u-u_k\|_1 \to 0.
    \label{eq:friedrich_cauchy_limit}
\end{equation}
We have proven Theorem~\ref*{thm:friedrich} for every one of these $u_k$, and 
since $s^2$ is independent of $k$, we can write 
\begin{equation*}
    \|u\|_0 \leq \|u_k\|_0 + \|u-u_k\|_0 \leq  s^2|u_k|_1 +  \|u-u_k\|_0 \leq s^2|u|_1 + s^2|u-u_k|_1 + \|u-u_k\|_0.
\end{equation*}
By (\ref*{eq:friedrich_cauchy_limit}), both of the last terms in the 
previous equation must tend to $0$ as $k \to \infty$, and we have the proof.
\end{bev}

This result can be generalized to $n$ order. 

\begin{kor}{Friedrichs' Inequality, $n$-order}
    Let $\Omega$ be contained in a hypercube with sidelength $s$. Then 
\begin{equation*}
    \| v \|_{k} \leq s |v|_{k+1} \quad \quad \forall v \in H_0^{k+1}(\Omega).
\end{equation*}\label{cor:friedrich_n}
\vspace{-8mm}
\end{kor}
\begin{bev}
We will prove this using induction, where the base case is Theorem~\ref*{thm:friedrich}.
We then take the basic step
\begin{equation*}
    \|u\|_{ H^k (\Omega)} \leq C_k |u|_k \quad \forall u \in H_0^k (\Omega).
\end{equation*}

Now the induction step. For $u\in C^{\infty}_0(\Omega)\in H^{k+1}(\Omega)\subset H^k(\Omega)$, we have
\begin{equation}
    |u|^2_k = \sum_{|\alpha| = k} \|\partial^\alpha u\|_{L_2(\Omega)}^2
\end{equation}
Thus $\partial^\alpha u \in H^1_0(\Omega)$. Then by the base case, we have
\begin{equation}
    \begin{split}
        \sum_{|\alpha|=k} \|\partial^\alpha u\|^2_{L_2(\Omega)} &\leq C_1^2 |\partial^\alpha u|_1^2 \quad \forall \alpha : |\alpha|=k \\
        &= C^2_1 \sum_{|\alpha|=k+1} |\partial^{\tilde{\alpha}} u |^2_0 = C_1^2 |u|_{k+1}
    \end{split}
\end{equation}  
We now write
\begin{equation}
    \|u\|^2_{H^{k+1}} \leq \tilde{C}^2 |u|^2_{k+1}.
\end{equation}
Thus
\begin{align*}
    \|u\|^2_{H^{k+1}} &=\|u\|^2_{H^k} + |u|^2_{k+1} \\
    &\leq \tilde{C}^2|u|^2_{k+1} + |u|^2_{k+1} \\
    &\leq (\tilde{C}^2 + 1)|u|^2_{k+1} \\
    &\leq C_{k+1}^2 |u|^2_{k+1}
\end{align*}
and the induction step is complete, finishing the proof.
\end{bev}