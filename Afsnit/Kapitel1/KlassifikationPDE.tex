
\begin{defn}{\quad}
	The general linear partial differential equation of second order in $n$ variables $x=(x_1,\ldots,x_n)$ has the form
	\begin{equation}
		-\sum_{ i, k = 1 }^n a_{ ik } (x) u_{ x_i x_k } + \sum_{ i = 1 }^n b_i (x) u_{ x_i } + c (x) u = f (x).
		\label{eq:generalPDE}
	\end{equation}
\end{defn}

\begin{defn}{Classification of a PDE}
	The equation (\ref{eq:generalPDE}) is classified as
		\begin{itemize}
			\item elliptic at $x$, if $A(x)$ is positive definite,
			\item hyperbolic at $x$, if $A(x)$ has one negative and $n-1$ eigenvalues and
			\item parabolic at $x$, if $A(x)$ is positive semidefinite, not positive definite, and rank$([A(x), b(x)])=0$.  
		\end{itemize}
	
	If, an equation holds the above conditions for all points of the domain, the equation is called elliptic, hyperbolic or parabolic respectively. 
	\end{defn}