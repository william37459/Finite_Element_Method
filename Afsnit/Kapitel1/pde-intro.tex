

\begin{equation}
	f(x) = c(x) u + \sum_{i=1}^{n}b_i(x)u_{x_{i}}
	- \sum_{i,k=1}^{n}a_{ik}(x)u_{x_ix_k}
	\label{eq:pde_order2_n}
\end{equation}

Explain $A(x)$ and $b(x)$.

\begin{defn}{Classification of a PDE}
	The Equation (\ref{eq:pde_order2_n}) is classified as follows:
	\begin{itemize}
		\item Elliptic at $x$, if $A(x)$ is positive definite
		\item Hyperbolic at $x$, if $A(x)$ has one negative and $n-1$ eigenvalues
		\item Parabolic at $x$, if $A(x)$ is positive semidefinite, not positive definite, and rank$([A(x), b(x)])=0$
	\end{itemize}

	If, for every element in the domain of a function, the function is elliptic, hyperbolic, or parabolic, the function itself is called it.
\end{defn}


\begin{defn}{Laplace Operator}
	The Laplace Operator $\triangle$
\end{defn}