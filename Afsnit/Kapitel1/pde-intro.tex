This chapter is based on \cite{Braess}.
The theory of Finite Element Theory (FEM) springs from a 
wish to solve Partial Differential Equations (PDE's) 
which may or may not have be analytically solvable. 
To do this, we introduce the kind of PDE's we will 
be working with, introducing the forms of the 
problems, and examining the spaces of solutions. 
In this chapter, we will always be working with 
equations with $n$ variables, and $\Omega$ will be 
an open subset of $\RR^n$ with a piecewise smooth boundary.
The PDE which we start with is a second order PDE, 
with the following form.
\begin{equation}
	f(x) = c(x) u + \sum_{i=1}^{n}b_i(x)u_{x_{i}}
	- \sum_{i,k=1}^{n}a_{ik}(x)u_{x_i x_k}\label{eq:pde_order2_n}
\end{equation}
The coefficients in (\ref*{eq:pde_order2_n}) can be 
organized nicely in a vector containing every $b_i$ 
in $b(x)$
and a matrix $A(x)$ containing every $a_{ik}$.
Assuming $u$ in (\ref*{eq:pde_order2_n}) is sufficiently 
smooth, then $u_{x_i x_k} =u_{x_k x_i} $, and $A(x)$ is 
symmetric. 
We use $A(x)$ to define the type of equation we are working 
with, and therefore to characterize the problem we are 
working with.
\begin{defn}{Classification of a PDE}
	The Equation (\ref{eq:pde_order2_n}) is classified as follows:
	\begin{itemize}
		\item Elliptic at $x$, if $A(x)$ is positive definite
		\item Hyperbolic at $x$, if $A(x)$ has one negative and $n-1$ eigenvalues
		\item Parabolic at $x$, if $A(x)$ is positive semidefinite, not positive definite, and rank$([A(x), b(x)])=0$
	\end{itemize}
	If, for every element in the domain of a function, the function is elliptic, hyperbolic, or parabolic, the function itself is called it.
\end{defn}
This chapter will work in several different spaces, with different norms and 
seminorms. 
These norms are available both in \cite{Braess} and in the List of Symbols. 
The spaces in which these norms are from should be obvious from context, but from 
time to time a space will be included in the subscript to specify exactly which 
space the norm is from.