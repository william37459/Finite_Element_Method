This chapter is based on~\cite{Braess}.
The theory of Finite Element Method (FEM) stems from a 
wish to solve Partial Differential Equations (PDE's),
which may or may not be analytically solvable. 
To do this we will introduce a specific kind of PDE, 
various types of 
problems, and examine the spaces containing solutions. 
The functions used in this chapter wil be of $n$ 
variables and $\Omega$ will be 
an open subset of $\RR^n$ with a piecewise smooth boundary.
The first PDE we introduce is a second order PDE and as 
such has the form
\begin{equation}
	 c(x) u + \sum_{i=1}^{n}b_i(x)u_{x_{i}}
	- \sum_{i,k=1}^{n}a_{ik}(x)u_{x_i x_k}\label{eq:pde_order2_n}
	= f(x).
\end{equation}

The coefficients in~\eqref{eq:pde_order2_n} can be 
organized in a vector containing every $b_i$ 
in $b(x)$
and a matrix $A(x)$ containing every $a_{ik}$.
Assuming $u$ in~\eqref{eq:pde_order2_n} is sufficiently 
smooth, then $u_{x_i x_k} =u_{x_k x_i} $, and we can 
assume without loss of generality $A(x)$ is 
symmetric. %? Passer dette bedre?
We use $A(x)$ to define the type of equation we are working 
with and therefore to characterize the problem we are 
working with.
\begin{defn}{Classification of a PDE}
	The Equation~\eqref{eq:pde_order2_n} is classified as follows:
	\begin{itemize}
		\item Elliptic at $x$, if $A(x)$ is positive definite
		\item Hyperbolic at $x$, if $A(x)$ has one negative and $n-1$ postive eigenvalues
		\item Parabolic at $x$, if $A(x)$ is positive semidefinite, not positive definite, and rank$([A(x), b(x)])=n$
	\end{itemize}
	If for a given equation the above conditions hold for all points of the domain, the equation is called elliptic, hyperbolic or parabolic respectively.\label{def:pde_classification}
\end{defn}


\begin{exmp}{\quad}
The PDE $c(x)u + u_{x_1x_1} + u_{x_2x_2} = f(x)$ is an example of a simple Elliptic PDE, since
\begin{equation}\label{eq:Basic matrix}
	A(x) = \begin{bmatrix}
		1 & 0\\
		0 & 1
	\end{bmatrix}, \quad
	b(x) = \begin{bmatrix}
		0\\
		0
	\end{bmatrix}
\end{equation}

Had entry $a_{11}$ been negative, the PDE would have been Hyperbolic, since the eigenvalues would be $-1$ and $1$. To convert~\eqref{eq:Basic matrix} to matrices for a Parabolic PDE, we can simply change $a_{11}$ to $0$, which results in $A(x)$ being positive semidefinite and $\text{rank}([A(x), b(x)]) = 2 = n$.
\end{exmp}
	

In this chapter we will work with several different spaces, and with different norms and 
seminorms. 
These norms are defined in~\cite{Braess} and in the List of Symbols. 
The spaces in which these norms are from should be obvious from context, but from 
time to time a space will be included in the subscript to specify exactly which 
space the norm is from.