\begin{thmx}{\quad}
   Let $\OO$  be bounded, have a piecewise smooth boundary, and satisfy the~\label{thm:neumann_solution_existence}
   cone condition. Let $f \in L_2( \OO)$ and $g\in L_2(\partial \OO)$.
   There exists a unique $u \in H^1(\OO)$ that solves the variational problem 
   \begin{equation*}
    J(v) = \frac{1}{2}a(v,v) - {(f,v)}_{0,\OO} - {(g,v)}_{0,\partial\OO} \to \min.
   \end{equation*}
   Also, $u \in C^2(\OO)\cap C^1(\bar{\OO})$ if and only if a classical solution 
   of 
   \begin{align}
    Lu &= f \quad \text{ in }\OO, \nonumber \\
    \sum_{i,k}  \mathbf{n}_i a_{ik} \partial_k u &= g \quad \text{ on } \partial \OO, \label{eq:neuman_condition_boundary}
   \end{align}
   exists, in which case these $2$ solutions are the same. Here $\mathbf{n}$ is outward 
   pointing normal on $\partial \OO$, defined almost everywhere.
\end{thmx}
\begin{bev}
   Obviously $ {(f,v)}_{0,\OO} - {(g,v)}_{0,\partial\OO}$ is a linear functional, 
   and since $a$ is $H^1$ elliptic, Theorem~\ref{thm:lax_milgram} gives us the
   uniqueness and existence of $u$.
   Corollary~\ref{cor:lax_milgram} gives us
   \begin{equation}
      a(u,v) = {(f,v)}_{0,\OO} - {(g,v)}_{0,\partial\OO} \quad \forall v \in H^1(\OO). \label{eq:neumann_lax}
   \end{equation}
   We now move on to the ``if and only if'' part of the theorem. Start by 
   assuming $u \in C^2(\OO)\cap C^1(\bar{\OO})$.
   If we look in the interior, and as such at $v \in H^1_0(\OO)$, we get $\gamma v=0$ 
   and~\eqref{eq:neumann_lax} simply becomes a a Dirichlet problem where 
   $u$ is used to define the boundary condition, see~\eqref{eq:b_v_problem_homogeneous}.
   On the interior we then have 
   \begin{equation}
    Lu = f \quad \text{ in }\OO, \label{eq:neumann_on_interior}
   \end{equation} 
   which is the first condition in~\eqref{eq:neuman_condition_boundary}.
   We now examine the boundary, and we assume $v\in H^1(\OO)$.
   In the proof of Theorem~\ref{thm:minimal_property}, we used Green's formula,
    \begin{equation} %? Skal dette slettes?
        \int_\Omega v\partial_i w dx = -\int_\Omega w \partial_i v dx + \int_{\partial \Omega} v w \mathbf{n}_i ds.
    \end{equation}
   In that proof we could assume the integral on the boundary to be zero, which 
   we cannot do here. However, with the same arguments regarding summing and substituting, 
   we get
   \begin{equation}
      a(v,v) - {(f,v)}_{0,\OO} - {(g,v)}_{0,\partial\OO} = 
      \int_\OO v[Lu - f] dx + \int_{\partial \OO}
     \left[ \sum_{i,k}  \mathbf{n}_i a_{ik} \partial_k u-g \right]v ds.
     \label{eq:existence_neumann_last_eq}
   \end{equation}
   Using~\eqref{eq:neumann_lax} and~\eqref{eq:neumann_on_interior},
   the last integral in~\eqref{eq:existence_neumann_last_eq} must be $0$.

   Suppose that for some $i$ and $k$ the function $v_0 = \mathbf{n}_i a_{ik} \partial_k u -g$ does 
   not vanish. Then $\int_{\partial\OO}v_0^2ds>0$, and by the density of 
   $C^1(\bar{\OO})$ in $C^0(\bar{\OO})$, there exists a $v\in C^1(\bar{\OO})$
   such that  $\int_{\partial\OO}v_0\cdot vds>0$, which cannot happen, due that 
   what we found using~\eqref{eq:existence_neumann_last_eq}. 
   We therefore have the second condition in~\eqref{eq:neuman_condition_boundary},
   and $u$ is a classical solution of this boundary problem.

   On the other hand, assuming $u$ is a classical solution and satisfy~\eqref{eq:neuman_condition_boundary},
   we see from~\eqref{eq:existence_neumann_last_eq} that $u$ also satisfy~\eqref{eq:neumann_lax}, and is therefore a solution to the variational 
   problem, and by Theorem~\ref{thm:lax_milgram} unique.
\end{bev}

