\section{Neumann Boundary Conditions}
We remind the reader of the bilinear form we are working with,
\begin{equation}
   a(u,v) = \int_\Omega \left(\sum_{i,k} a_{ik}\partial_i u\partial_k v+a_0uv\right)dx.
\end{equation}
We now require $a_0(x)$ in the Elliptic PDE, see~\ref{def:pde_classification}, to be bounded from below by a positive number (and so also in $a$). 
Inequality~\eqref{eq:alpha_ellipticity} will also be true for smaller $\alpha$, so if 
necessary we can reduce it, and assume 
\begin{equation*}
  a_0(x) \geq \alpha >0 \quad \forall x \in \OO.
\end{equation*}
This gives us the lower bound on $a$ as 
\begin{equation*}
   a(v,v)  \geq \alpha |v|^2_1 + \alpha \|v\|^2_0 = \alpha \|v\|^2_1 \quad \forall v \in H^1(\OO).
\end{equation*}
We can therefore assume from this point on that $a$ is $H^1$ elliptic.

\begin{defn}{Neumann Boundary condition}
    Let $\frac{\partial u}{\partial \mathbf{n}}= \mathbf{n}\cdot \nabla u$ be the normal derivative. Then the Neumann boundary condition is given by
    \begin{equation}
        \frac{\partial u}{\partial \mathbf{n}}  = g\quad \text{on } \partial \Omega.
    \end{equation}
\end{defn}
With this definition, we can start looking at the existence of solutions to problems which are defined using this type of condition.
\begin{thmx}{\quad}
   Let $\OO$  be bounded, have a piecewise smooth boundary, and satisfy the~\label{thm:neumann_solution_existence}
   cone condition. Let $f \in L_2( \OO)$ and $g\in L_2(\partial \OO)$.
   There exists a unique $u \in H^1(\OO)$ that solves the variational problem 
   \begin{equation*}
    J(v) = \frac{1}{2}a(v,v) - {(f,v)}_{0,\OO} - {(g,v)}_{0,\partial\OO} \to \min.
   \end{equation*}
   Also, $u \in C^2(\OO)\cap C^1(\bar{\OO})$ if and only if a classical solution 
   of 
   \begin{align}
    Lu &= f \quad \text{ in }\OO, \nonumber \\
    \sum_{i,k}  \mathbf{n}_i a_{ik} \partial_k u &= g \quad \text{ on } \partial \OO, \label{eq:neuman_condition_boundary}
   \end{align}
   exists, in which case these $2$ solutions are the same. Here $\mathbf{n}$ is outward 
   pointing normal on $\partial \OO$, defined almost everywhere.
\end{thmx}
\begin{bev}
   Due to the Trace Theorem,~\ref{thm:trace}, the functional ${(g,v)}_{0,\partial \OO}$ is bounded,
   and thus $ {(f,v)}_{0,\OO} - {(g,v)}_{0,\partial\OO}$ is a bounded linear functional, 
   and since $a$ is $H^1$ elliptic, Theorem~\ref{thm:lax_milgram} gives us the
   uniqueness and existence of $u$.
   Corollary~\ref{cor:lax_milgram} gives us
   \begin{equation}
      a(u,v) = {(f,v)}_{0,\OO} + {(g,v)}_{0,\partial\OO} \quad \forall v \in H^1(\OO). \label{eq:neumann_lax}
   \end{equation}
   We now move on to the ``if and only if'' part of the theorem. Start by 
   assuming $u \in C^2(\OO)\cap C^1(\bar{\OO})$.
   If we look in the interior, and as such at $v \in H^1_0(\OO)$, we get $\gamma (v)=0$ 
   and~\eqref{eq:neumann_lax} simply becomes a a Dirichlet problem where 
   $u$ is used to define the boundary condition, see~\eqref{eq:b_v_problem_homogeneous}.
   On the interior we then have 
   \begin{equation}
    Lu = f \quad \text{ in }\OO, \label{eq:neumann_on_interior}
   \end{equation} 
   which is the first condition in~\eqref{eq:neuman_condition_boundary}.
   We now examine the boundary, and we assume $v\in H^1(\OO)$.
   In the proof of Theorem~\ref{thm:minimal_property}, we used Green's formula,
    \begin{equation} %? Skal dette slettes?
        \int_\Omega v\partial_i w dx = -\int_\Omega w \partial_i v dx + \int_{\partial \Omega} v w \mathbf{n}_i ds.
    \end{equation}
   In that proof we could assume the integral on the boundary to be zero, which 
   we cannot do here. However, with the same arguments regarding summing and substituting, 
   we get
   \begin{equation}
      a(u,v) - {(f,v)}_{0,\OO} - {(g,v)}_{0,\partial\OO} = 
      \int_\OO v[Lu - f] dx + \int_{\partial \OO}
     \left[ \sum_{i,k}  \mathbf{n}_i a_{ik} \partial_k u-g \right]v ds.
     \label{eq:existence_neumann_last_eq}
   \end{equation}
   Using~\eqref{eq:neumann_lax} and~\eqref{eq:neumann_on_interior},
   the last integral in~\eqref{eq:existence_neumann_last_eq} must be $0$.

   Suppose that for some $i$ and $k$ the function $v_0 = \mathbf{n}_i a_{ik} \partial_k u -g$ does 
   not vanish on $\partial \OO$. 
   Then $\int_{\partial\OO}v_0^2ds>0$, and by the density of 
   $C^1(\bar{\OO})$ in $C^0(\bar{\OO})$, there exists a $v\in C^1(\bar{\OO})$
   such that  $\int_{\partial\OO}v_0\cdot vds>0$, which cannot happen, due that 
   what we found using~\eqref{eq:existence_neumann_last_eq}. 
   We therefore have the second condition in~\eqref{eq:neuman_condition_boundary},
   and $u$ is a classical solution of this boundary problem.

   On the other hand, assuming $u$ is a classical solution and satisfy~\eqref{eq:neuman_condition_boundary},
   we see from~\eqref{eq:existence_neumann_last_eq} that $u$ also satisfy~\eqref{eq:neumann_lax}, and is therefore a solution to the variational 
   problem, and by Theorem~\ref{thm:lax_milgram} unique.
\end{bev}
Not all Neumann problems fulfill the assumption that $a_0(x)>0$. For example in the 
Neumann Poisson problem,  
\begin{align}
\label{eq:Neumann_Poisson}
\begin{split}
    -\Delta u &= f \quad \text{in } \Omega, \\
    \frac{\partial u}{\partial \mathbf{n}} &= g \quad \text{on } \partial \Omega.
\end{split}
\end{align}
one can find $a_0(x)=0$. We remedy that by constructing a subspace $V$.
Due to the fact that~\eqref{eq:Neumann_Poisson} only contains derivatives with respect to $u$,
 additive constants independent of $u$ are unimportant.
Thus if $u$ satisfies~\eqref{eq:Neumann_Poisson}, then $u + c$ also satisfies~\eqref{eq:Neumann_Poisson} for any constant $c$.
This means that the solution to~\eqref{eq:Neumann_Poisson} is only unique up to a constant,
 which extends to other equations with Neumann boundary conditions.

It turns out that we can formulate the weak form of this problem by restricting to the subspace $V=\{v\in H^1(\Omega):\int_{\Omega}vdx=0\}$.
Define $a(u,v) = \int_\OO \nabla u \cdot \nabla v dx$. Examining a version of 
the Friedrich inequality, Theorem~\ref{thm:friedrich}, which can be found on 
page 46 of~\cite{Braess}, $a$ can be found to be $V$-elliptic.
We now want to assert that every classical solution to the variational problem in Theorem~\ref{thm:neumann_solution_existence} satisfies~\eqref{eq:Neumann_Poisson}.
To do this, we let $w=\nabla u$. Thus~\eqref{eq:Neumann_Poisson} can be written as
\begin{align}\label{eq:Neumann_Poisson_weak}
\begin{split}
    -\text{div } w &= f \quad \text{in } \Omega, \\
    \mathbf{n}^T w &= g \quad \text{on } \partial \OO.
\end{split}    
\end{align}
From the Gauss Integral Theorem, we have that
\[\int_{\Omega} \text{div } w dx = \int_{\partial \OO} \mathbf{n}^T w ds.\]
Then from~\eqref{eq:Neumann_Poisson_weak} we get
\begin{align}
\begin{split}
%\int_{\Omega} \text{div } w dx &= \int_{\partial \OO} \mathbf{n}^Tw ds \\ 
\int_{\Omega} -f dx &= \int_{\partial \OO} g ds.
\end{split}
\end{align}
The integral used in $V$ is a bounded linear functional, and thus 
$V$ is a closed and linear subspace of a Hilbert space, 
V is complete with respect to the norm, and is therefore also a Hilbert space. 
We can then use Corollary~\ref{cor:lax_milgram} 
 and get $u\in V$, where
\begin{equation}\label{eq:Lax-Milgram_Neumann}
  a(u,v) = {(f,v)}_{0,\Omega} + {(g,v)}_{0,\partial \OO} \quad \forall v\in V.  
\end{equation}
Equation~\eqref{eq:Neumann_Poisson} has now been shown to be true for 
every classical solution of the variational problem for $u\in V$. However, for some 
$f\in H^1(\OO)\setminus V$, we can use the following construction for 
the constant $c$:
\begin{equation*}
    \forall f \in H^1(\OO): \,\, c = \frac{\int_\OO f(x) dx }{\int_\OO 1 dx}.
\end{equation*}
Define $v(x) = f(x) - c$, and then 
\begin{equation*}
    \int_\OO v(x)dx = \int_\OO f(x) dx - c\int_\OO 1 dx = \int_\OO f(x) dx - \frac{\int_\OO f(x) dx }{\int_\OO 1 dx} \int_\OO 1 dx = 0,
\end{equation*}
so $v \in V$. For every $u\in H^1(\OO)\setminus V$ we can find a fitting 
constant such that $u+c \in V$. Every classical solution of the 
variational problem corresponding to a Neumann Boundary condition therefore 
satisfies~\eqref{eq:Neumann_Poisson}, and~\eqref{eq:Neumann_Poisson} becomes the weak formulation.

The end result is we look for the solution $u\in V$, and use test functions 
$v\in H^1(\OO)$.