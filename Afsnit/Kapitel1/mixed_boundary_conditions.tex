A word on Dirichlet boundary conditions and Neumann boundary 
conditions - these conditions are not mutually exclusive.
It is possible to have a problem where on some parts of the boundary, 
a constant condition is required, and on others there are not.
% ? Dette er måske et dumt eksempel.

Take for example the flow of air in a room with open windows. 
Assume you have a (very powerful) vacuum cleaner,
 and point in up in the air at some fixed point. 
We know the constant flow of air around the mouth of the vacuum cleaner, which 
gives us a Dirichlet condition there, but the flow of the air through the 
window is not constant, and are better described using Neumann conditions.

These kind of problems lead looking for a solution in a space $V$ such that 
$H^1_0(\OO) \subset V \subset H^1(\OO)$, with any other conditions used such 
that $V$ is an appropriate space.