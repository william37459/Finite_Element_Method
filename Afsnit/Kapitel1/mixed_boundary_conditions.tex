\begin{exmp}{Mixed conditions}
Consider the following problem:
In a room with two open windows, and a vacuum cleaner placed in the middle of the room poiting at a fixed point. The wind flowing through the windows can be described using a Neumann boundary condition, while the flow around the vacuum cleaner would result in a Dirichlet boundary condition.
\end{exmp}

\iffalse\\
A word on Dirichlet boundary conditions and Neumann boundary 
conditions-these conditions are not mutually exclusive.
It is possible to have a problem where on some parts of the boundary, 
a constant condition is required, and on other parts not.
% ? Dette er måske et dumt eksempel.

Take for example the flow of air in a room with open windows. 
Assume you have a (very powerful) vacuum cleaner,
 and point it up in the air at some fixed point. 
We know the constant flow of air around the mouth of the vacuum cleaner, which 
gives us a Dirichlet condition there, but the flow of air through the 
windows is not constant, and are better described using Neumann conditions.
\fi
