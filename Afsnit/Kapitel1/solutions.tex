\section{Variational Problems and Solutions}
The goal of this section is to start developing the mathematical 
precision to know of the existence of solutions to a PDE, and how to find 
them. 
To do this, we start with a general abstract theorem on how solutions 
behave in certain problems, then we look at a connection between 
classical solutions and solutions of variational problems, a 
theorem regarding uniqueness of solutions, and existence of 
weak solutions.


\begin{thmx}{Characterization Theorem}
\label{thm:charac_theorem}
    Let $V$ be a linear subspace, such that
        \[
            a: VxV \rightarrow \mathbb{R}
        \]
    is a symmetric positive bilinear form, then $a(v,v)>0$ for all $v \in V$, $v \neq 0$. In addition, let
        \[
            \ell: V \rightarrow \mathbb{R}
        \]
    be a linear functional. Then the quantity

        \[
            J(v):=\frac{1}{2} a(v,v) - \langle\ell,v\rangle 
        \]
    optains its minimum over $V$ at $u$ if and only if
        \begin{equation}
        \label{eq:charac}
            a(u,v) = \ell(v) \quad \text{for all } v \in V.
        \end{equation}
    There is at most one solution to~\eqref{eq:charac}.
\end{thmx}

\begin{proof}

    For $u,v \in V$ and $t \in \mathbb{R}$ we have, that
    \begin{align}
        J(u+tv) &= \frac{1}{2} a(u+tv,u+tv) - \langle\ell,u+tv\rangle \nonumber \\
        &= \frac{1}{2} \left( a(u,u) + a(tv,tv) + 2a(u,tv) \right) - \left( \langle \ell,u\rangle + \langle\ell,tv\rangle \right) \nonumber \\
        &= J(u) + t\left( a(u,v) - \langle\ell,v\rangle \right) + \frac{1}{2}t^2a(v,v). \label{eq:charac_proof_J(u+tv)}
    \end{align}

    If $u \in V$ satisfies~\eqref{eq:charac} and $t=1$, then from~\eqref{eq:charac_proof_J(u+tv)} we have, that
    \begin{alignat}{2}
        J(u+v) &= J(u) + \frac{1}{2}a(v,v) \quad &&\text{for all } v\in V  \nonumber \\
        &> J(u) \quad &&\text{for } v \neq 0.
    \end{alignat}
    Thus $u$ is a ubnique minimal point. 

    To prove the opposite way, we assume that $J$ has a minimum at $u$, then for every $v\in V$, the derirative of the function $t \mapsto J(u+tv)$ must vanish at $t=0$. 
    By~\eqref{eq:charac_proof_J(u+tv)} the derirative of $J(u+tv)$ with respect to $t$ can be found using, that
    \begin{align*}
          J(u+(t+h)v) &= J(u) + (t+h) (a(u,v) - \langle \ell,v\rangle) + \frac{1}{2}(t+h)^2 a(v,v) \\
          &= J(u+tv) + h(a(u,v) - \langle \ell,v\rangle) + \frac{1}{2}((h^2+2th)a(v,v)),
    \end{align*}
    thus the derirative is $a(u,v) - \langle \ell,v\rangle$, and~\eqref{eq:charac} then follows.
\end{proof}
Theorem~\ref{thm:charac_theorem} does not guarantee the existence 
of solutions, it only characterizes them. In Theorem~\ref{thm:charac_theorem} we 
also only make assumptions on the linearity of the space. 
It is possible to set up a variational problem such that $J$ does 
not attain its minimum.
To engage with the existence of solutions, we make more assumptions on 
the space in which we work. To do this we work with Hilbert spaces. 
We start by specifying the bilinear form in Theorem~\ref{thm:charac_theorem}.
\begin{defn}{\quad}
   Let $H$ be a Hilbert space. A bilinear form $a : H \times H \to \RR$ is 
   called continuous if there exists some $C > 0$ such that 
   \begin{equation}
    |a(u,v)| \leq C \|u\|\, \|v\| \quad \forall u,v \in H.
   \end{equation} 
   If a bilinear form $a$ is symmetric and continuous, and there exists som $\alpha >0$ such that 
   \begin{equation*}\label{eq:elliptic}
    a(v,v) \geq \alpha \|v\|^2 \quad \forall v \in H,
   \end{equation*}
   $a$ is called elliptic.
 \end{defn}
From this point forward $a$ will be referring to the bilinear form, given by
\begin{equation}
   a(u,v) = \int_\Omega \left[\sum_{i,k} a_{ik}\partial_iu\partial_kv+a_0uv\right]dx,
\end{equation}
unless otherwise stated.

We now use Theorem~\ref{thm:charac_theorem} to show a link between classical 
solutions and solutions of appropriate variational problems.
\begin{thmx}{Minimal Property}
    Every classical solution of the boundary-value problem, given by
\begin{align}
    -\sum_{i,k} \partial_i (a_{ik}\partial_k u) + a_0 u &= f \quad \text{in } \Omega  \\
    u &= 0 \quad \text{on } \partial \Omega,
\end{align}
    is a solution to the variational problem, given by
    \[
        J(v)=\int_\Omega \left [\frac{1}{2}\sum_{i,k} a_{ik} \partial_i v\partial_k v + \frac{1}{2} a_0 v^2 -fv\right ]dx \longrightarrow \min,
    \]
    among all functions in $C^2(\Omega)\cap C^0(\bar{\Omega})$ with zero boundary values. 
\end{thmx}
\begin{proof}
    We start off by applying Green's formula
    \begin{equation}
    \label{eq:Greens_formula}
        \int_\Omega v\partial_i w dx = -\int_\Omega w \partial_i v dx + \int_{\partial \Omega} v w \vec{n}_i ds.
    \end{equation}
    Here we assume $v$ and $w$ to be $C^1(\OO)$ functions, and $\vec{n}_i$ is the $i$'th component of the outward-pointing normal $\vec{n}$.
    Now if we insert $w=a_{ik}\partial_k u$ in~\eqref{eq:Greens_formula}, we get
    \begin{equation}
    \label{eq:Greens_formula_inserted}
        \int_\Omega v\partial_i (a_{ik} \partial_k u) dx = -\int_\Omega a_{ik} \partial_i v v\partial_k u dx,
    \end{equation}
    given that $v=0$ on $\partial \Omega$.
    Now let 
    \begin{equation}
    \label{eq:a(u,v)}
        a(u,v) = \int_\Omega \left[\sum_{i,k} a_{ik} \partial_i u \partial_k v +a_0 uv \right]dx
    \end{equation}
    and
    \begin{equation}
    \label{eq:l(v)}
        \ell(v) = \int_\Omega fv dx.
    \end{equation}
    Then by summing~\eqref{eq:Greens_formula_inserted} over $i$ and $k$ we get that for every $v\in C^1(\Omega) \cap C(\bar{\Omega})$ with $v=0$ on $\partial \Omega$ we have
    \begin{align}
        a(u,v) - \ell(v) &= \int_\Omega v\left[ -\sum_{i,k} \partial_i (a_{ik} \partial_k u) + a_0 u - f \right] dx \label{eq:min_prop_variational_problem}\\
        &= \int_\Omega v [Lu - f] dx \nonumber \\
        &= 0, \nonumber
    \end{align}
    given that $Lu = f$. This property holds if $u$ is a classical solution.
     The minimal property is then implied by Theorem~\ref{thm:charac_theorem}
\end{proof}
The same kind of proof can be used to show that a solution $u$ to 
(\ref{eq:min_prop_variational_problem}) which fulfills 
$u\in C^2(\OO) \cap C^0(\OO)$ is also a classical solution. 
This gives a connection between classical solutions and solutions of 
variational problems. We now proceed to examine the existence of a 
solution.
\begin{thmx}{Lax-Milgram Theorem}
Let $V$ be a closed, convex, non-empty set in a Hilbert space $H$, and let $a:H \times H \rightarrow \mathbb{R}$ be an elliptic bilinear form. Then for every $\ell\in H'$, the variational problem given by
\[
    J(v)=\frac{1}{2} a(v,v) - \ell(v) \longrightarrow \min    
\]
has a unique solution in $V$.\label{thm:lax_milgram}
\end{thmx}

\begin{bev}
    $J$ is bounded from below, since
    \begin{align*}
        J(v) &\geq \frac{1}{2} \alpha \|v\|^2 - \|\ell\| \, \|v\|\\
        &= \frac{1}{2\alpha} {(\alpha \|v\|-\|\ell\|)}^2 - \frac{\|\ell\|^2}{2\alpha} \\
        &\geq - \frac{\|\ell\|^2}{2\alpha}.
    \end{align*}
    We then let $c_1 = \inf\{J(v); v \in V\}$ and ${\{v_n \}}_{n=1}^\infty$ be a minimizing sequence. Then we get 
    \begin{align*}
        \alpha \|v_n-v_m\|^2 &\leq a(v_n-v_m,v_n-v_m) \\
        &= 2a(v_n,v_n) + 2a(v_m,v_m) - a (v_n+v_m,v_n+v_m) \\
        &= 4J(v_n) + 4J(v_m) - 8J(\frac{v_m+v_n}{2}) \\
        &\leq 4J(v_n) + 4J(v_m) - 8c_1. 
    \end{align*}
    By the convexity assumption on $V$, we get that $\frac{1}{2}(v_n + v_m) \in V$, and from that we get the last 
    inequality.
    We now get that ${\{v_n \}}_{n=1}^\infty$ is a Cauchy sequence in $H$ and $u = \lim_{n\rightarrow \infty}v_n$ exists given the fact that $J(v_n),J(v_m)\rightarrow c_1$ implies $\|v_n - v_m\| \rightarrow 0$ for $n,m\rightarrow \infty$.  

    Since $J$ is continuous,
    we see $J(u) = \lim_{n\rightarrow \infty} J(v_n) = \inf_{v\in V} J(v)$.
     Moreover, we also have that $u\in V$, because we assumed $V$ to be closed. 
    Lastly, we need to show that the solution is unique.
     To do that, we assume both $u_1$ and $u_2$ to be solutions of the variational problem. 
    Then we set up a minimizing sequence $u_1,u_2,u_1,u_2,\ldots$, which we know to be a Cauchy sequence from earlier.
     This implies $u_1 = u_2$.
\end{bev}
The Hilbert spaces we are going to be working with are all going to be 
vector spaces, and thus convex by definition. 
We can therefore use Theorem \ref{thm:lax_milgram} without further considerations 
on our spaces.
We can therefore make use of the following corollary to unify our approach.
\begin{kor}{\quad}
   Assume the same as in Theorem \ref{thm:lax_milgram}, except $V=H$. Then 
   the solution $u$ is given by 
   \label{cor:lax_milgram}
   \begin{equation*}
    a(u,v) = \ell(v).
   \end{equation*}
   \vspace{-9mm}
\end{kor}
\begin{bev}
    The proof follows closely the proof for Theorem \ref{thm:lax_milgram}. 
    Simply replace $V$ with $H$, to find the existence and uniqueness of the 
    solution, and use Theorem \ref{thm:charac_theorem} to find the form, as 
    shown in the corollary.
\end{bev}
Theorem \ref{thm:minimal_property}, Theorem \ref{thm:lax_milgram} and Corollary 
\ref{cor:lax_milgram} together shows us, that solving a Dirichlet problem is 
the same as finding the minimum of the corresponding variational problem. 
We can use this in our approach, both theoretically, and computationally.

When working with PDE's the equation shown in Corollary \ref{cor:lax_milgram} 
is called the variational formulation of a boundary condition problem. 
The general gist is to find some trial function $u$ and examine whether this 
is a solution to our problem, by using some test function $v$. 
As in Corollary \ref{cor:lax_milgram}, all the parts involving the trail function 
is grouped into one part, $a$, and the parts only involving the test functions into 
another, $\ell$.
This changes the problem to something easier to approximate and widens the 
space in which we look for solutions.
%https://fab.cba.mit.edu/classes/864.14/text/variational.pdf
%https://www.ljll.fr/ledret/M1English/M1ApproxPDE_Chapter3.pdf
%https://hplgit.github.io/num-methods-for-PDEs/doc/pub/varform/sphinx/._main_varform001.html#abstract-notation-for-variational-formulations

After proving these statements on qualities of a solution, we can now move 
on to discussing the existence of these solutions. We start by looking at 
sufficiently well-defined solutions. 
% TODO Ændre tekst - har vist noget med eksistens og entydighed, kigger på rum af løsningerne
\begin{defn}{\quad}
    The Dirichlet problem with homogeneous boundary conditions,
     (\ref{eq:b_v_problem_homogeneous}), has a weak solution $u\in H_0^1(\OO)$
     if 
     \begin{equation*}
        a(u,v) = (\tilde{f},v)_0 \quad \forall v \in H_0^1(\OO).
     \end{equation*}
     Here $a(u,v)$ is the bilinear from from Equation (\ref{eq:a(u,v)}), 
     and $(\tilde{f},v)_0$ is a linear bounded functional provided $\tilde{f}\in L^2$, and it can therefore be used as $\ell(v)$ in equation (\ref{eq:l(v)}).
\end{defn}


\begin{thmx}{Existence theorem}
    Let $L$ be a second order uniformly elliptic partial differential operator from~\eqref{eq:Lu}. Then the homogeneous Dirichlet problem,
     (\ref{eq:b_v_problem_homogeneous}), always has a weak solution in $H_0^1(\OO)$. It is a minimum of the variational problem
    \begin{equation}
        \frac{1}{2} a(v,v) - {(f, v)}_0 \rightarrow \text{min}
    \end{equation}
    over $H_0^1(\OO)$.\label{thm:existence_hom_dirichlet}
\end{thmx}

\begin{bev}
    Let $c = \sup\{\left| a_{ik}(x) \right| \; | \; x \in \OO, 1 \leq i,k \leq n\}$. Then by the Cauchy-Schwarz inequality we get
    \begin{align}
    \begin{split}
        \left| \sum_{i,k} \int a_{ik} \partial_i u \partial_k v dx \right| &\leq c \sum_{i,k} \int  \left|  \partial_i u \partial_k v dx\right|\\
        &\leq c \sum_{i,k} {\left[ \int {\left( \partial_i u \right)}^2 dx \int {\left( \partial_k v \right)}^2 dx  \right]}^{1/2} \\
        &\leq C {\left| u \right|}_1 {\left| v \right|}_1 \label{eq:Cuv}
    \end{split}
    \end{align}
    where $C=cn^2$. 
    We also assume $C\geq \sup\{|a_0(x)| \; | \; x\in \Omega\}$ giving us
    \begin{equation}
        \left|\int a_0 u v dx \right| \leq C \int |u v| dx \leq C \|u\|_0 \|v\|_0.  
        \label{eq:estimate_a_0}   
    \end{equation}
    Then by~\eqref{eq:Cuv} and~\eqref{eq:estimate_a_0} we get continuity of $a$.
    Furthermore the uniform ellipticity implies the pointwise estimate %TODO Hvorfor gør den det?
    \[
        \sum_{i,k} a_{ik} \partial_i v \partial_k v \geq \alpha \sum_i {\left( \partial_i v \right)}^2,
    \]
    for $\text{C}^1(\OO)$ functions. Integrating both sides and using $\text{a}_0 \geq 0$ leads to %TODO Er det fordi man fjerner noget i integralet?
    \begin{equation}
        a(v,v) \geq \alpha \sum_i \int_\OO {\left(\partial_i v\right)}^2 dx = \alpha |v|^2_1, \quad \forall v \in H^1(\OO).
        \label{eq:alpha_ellipticity}
    \end{equation}
    %TODO ref Friedrich equivalent
    We know from Friedrichs' Inequality $|\cdot|_1$ and $\| \cdot \|_1$ are equivalent norms on $H_0^1$,
     resulting in $a$ being an $H^1$-elliptic bilinear form on $H_0^1(\OO)$.
    From Theorem~\ref{thm:lax_milgram} there exists a unique weak solution which is also a solution of the variational problem.
\end{bev}


We can use Theorem \ref{thm:existence_hom_dirichlet} to examine the inhomogeneous Dirichlet problem, 
see (\ref{eq:b_v_problem}). From (\ref{eq:b_v_problem_homogeneous}), let 
$\tilde{u} \in C^2(\OO)\cap C^0(\bar{\OO}) \cap H^1(\OO)$ be a function such that 
\begin{equation*}
   \tilde{u} = g \quad \text{on } \partial \OO.
\end{equation*}
To find a weak solution $w \in H_0^1(\OO)$, we use Theorem \ref{thm:existence_hom_dirichlet} and the 
fact that $(L\tilde{u},v) = a(\tilde{u},v)$, to solve (\ref{eq:b_v_problem}), and
we find $u \in H^1(\OO)$ such that 
\begin{align*}
   a(u,v) = (f,v)_0 \quad \forall v &\in H_0^1(\OO) \\
   u-\tilde{u} & \in H_0^1(\OO).
\end{align*}
%TODO Hvorfor u i H^1? Density assumptions?