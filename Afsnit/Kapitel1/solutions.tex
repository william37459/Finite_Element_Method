\section{Variational Problems and Solutions}
The goal of this section is to start developing the mathematical 
precision to know of the existence of solutions to a PDE, and find 
them. 
To do this, we start with a general abstract theorem on how solutions 
behave in certain problems, then we look at a connection between 
classical solutions and solutions of variational problems, then a 
theorem regarding uniqueness of solutions, and then existence of 
weak solutions.


\begin{thmx}{Characterization Theorem}
    Let $V$ be a linear subspace, such that
        \(
            a: V \times V \rightarrow \mathbb{R}
        \) and
   % is a symmetric positive bilinear form
    %, then $a(v,v)>0$ for all $v \in V$, $v \neq 0$. In addition,
     let
        \(
            \ell: V \rightarrow \mathbb{R}
        \)
    be a linear functional. Then the quantity
        \[
            J(v):=\frac{1}{2} a(v,v) - \ell(v)
        \]
    obtains its minimum over $V$ at $u$ if and only if
        \begin{equation}
            a(u,v) = \ell(v) \quad \text{for all } v \in V.
        \label{eq:charac}
        \end{equation}
    There is at most one solution to~\eqref{eq:charac}.\label{thm:charac_theorem}
\end{thmx}

\begin{proof}
    For $u,v \in V$ and $t \in \mathbb{R}$ we have, that
    \begin{align}
        J(u+tv) &= \frac{1}{2} a(u+tv,u+tv) - \ell(u+tv) \nonumber \\
        &= \frac{1}{2} \left( a(u,u) + a(tv,tv) + 2a(u,tv) \right) - \left(  \ell(u) + \ell(tv)\right) \nonumber \\
        &= J(u) + t\left( a(u,v) - \ell(v) \right) + \frac{1}{2}t^2a(v,v). \label{eq:charac_proof_J(u+tv)}
    \end{align}
    If $u \in V$ satisfies~\eqref{eq:charac} and $t=1$, then from~\eqref{eq:charac_proof_J(u+tv)} we have, that
    \begin{alignat}{2}
        J(u+v) &= J(u) + \frac{1}{2}a(v,v) \quad &&\text{for all } v\in V  \nonumber \\
        &> J(u) \quad &&\text{for } v \neq 0.
    \end{alignat}
    Thus $u$ is a unique minimal point. 
    To prove the opposite way, we assume that $J$ has a minimum at $u$.
    Then for every $v\in V$, the function $f(t)= J(u+tv)$ must fulfill the condition
    \begin{equation*}
        \frac{d}{dt}f(0) = 0,
    \end{equation*}
    since $J$ has a minimum at $u$, and as such any $t>0$ must increase the 
    value of $J$.
    This derivative can be found using little-o notation, like so:
    \begin{align*}
          f(t+h) = J(u+(t+h)v) &= J(u) + (t+h) (a(u,v) - \ell(v)) + \frac{1}{2}{(t+h)}^2 a(v,v) \\
          &= J(u+tv) + h(a(u,v) - \ell(v)) + \frac{1}{2}((h^2+2th)a(v,v)).
    \end{align*}
    Thus the derivative is $a(u,v) - \ell(v)$, and~\eqref{eq:charac} then follows.
\end{proof}
Theorem~\ref{thm:charac_theorem} does not guarantee the existence 
of solutions, only their shape. In Theorem~\ref{thm:charac_theorem} we 
also only make assumptions on the linearity of the space. 
It is possible to setup a variational problem such that $J$ does 
not obtain its minimum.
To engage with the existence of solutions, we make more assumptions on 
the space in which we work. To do this we work with Hilbert spaces. 
We start by specifying the bilinear form in Theorem~\ref{thm:charac_theorem} 
a bit.
\begin{defn}{\quad}
   Let $H$ be a Hilbert space. A bilinear form $a : H \times H \mapsto \RR$ is 
   called continuous if there exists some $C > 0$ such that 
   \begin{equation}
    |a(u,v)| \leq C \|u\|\, \|v\| \quad \forall u,v \in H.
    \label{eq:testte}
   \end{equation} 
   If the bilinear form $a$ is symmetric and continuous, it is called 
   elliptic if there exists som $\alpha >0$ such that 
   \begin{equation*}
    a(v,v) \geq \alpha \|v\|^2 \quad \forall v \in H.
   \end{equation*}
\end{defn}
We now use Theorem~\ref{thm:charac_theorem} to show a link between classical 
solutions and solutions of appropriate variational problems.
\begin{thmx}{Minimal Property}
    Every classical solution of the boundary-value problem, given by
\begin{align}
    -\sum_{i,k} \partial_i (a_{ik}\partial_k u) + a_0 u &= f \quad \text{in } \Omega  \\
    u &= 0 \quad \text{on } \partial \Omega,
\end{align}
    is a solution to the variational problem given by
    \[
        J(v)=\int_\Omega [\frac{1}{2}\sum_{i,k} a_{ik} \partial_i v\partial_k v + \frac{1}{2} a_0 v^2 -fv]dx \longrightarrow \min
    \]
    among all functions in $C^2(\Omega)\cap C^0(\bar{\Omega})$ with zero boundary values. 
\end{thmx}
\begin{proof}
    We start off by applying Green's formula
    \begin{equation}
    \label{eq:Greens_formula}
        \int_\Omega v\partial_i w dx = -\int_\Omega w \partial_i v dx + \int_{\partial \Omega} v w \vec{n}_i ds.
    \end{equation}
    Here we assume $v$ and $w$ to be $C^1(\OO)$ functions, and $\vec{n}_i$ is the $i$'th component of the outward-pointing normal $n$.
    Now if we insert $w=a_{ik}\partial_k u$ in~\eqref{eq:Greens_formula}, we get
    \begin{equation}
    \label{eq:Greens_formula_inserted}
        \int_\Omega v\partial_i (a_{ik} \partial_k u) dx = -\int_\Omega a_{ik} \partial_i v v\partial_k u dx,
    \end{equation}
    given that $v=0$ on $\partial \Omega$.
    Now let 
    \begin{equation}
    \label{eq:a(u,v)}
        a(u,v) = \int_\Omega \left[\sum_{i,k} a_{ik} \partial_i u \partial_k v +a_0 uv \right]dx
    \end{equation}
    and
    \begin{equation}
    \label{eq:l(v)}
        \ell(v) = \int_\Omega fv dx.
    \end{equation}
    Then by summing~\eqref{eq:Greens_formula_inserted} over $i$ and $k$ we get that for every $v\in C^1(\Omega) \cap C(\bar{\Omega})$ with $v=0$ on $\partial \Omega$ we have, that 
    \begin{align}
        a(u,v) - \ell(v) &= \int_\Omega v\left[ -\sum_{i,k} \partial_i (a_{ik} \partial_k u) + a_0 u - f \right] dx \label{eq:min_prop_variational_problem}\\
        &= \int_\Omega v [Lu - f] dx \nonumber \\
        &= 0 \nonumber
    \end{align}
    given that $Lu = f$. This property holds if $u$ is a classical solution.
     The minimal property is then implied by Theorem~\ref{thm:charac_theorem}
\end{proof}
The same kind of proof can be used to show, that a solution $u$ to 
(\ref{eq:min_prop_variational_problem}) which fulfills 
$u\in C^2(\OO) \cap C^0(\OO)$ is also a classical solution. 
This gives a connection between classical solutions and solutions of 
variational problems. We now proceed to examine the existence of a 
solution.
\begin{thmx}{The Lax-Milgram Theorem}
    
\end{thmx}