\section{Hilbert Spaces}
When working with variational problems and the finite 
element method, the end goal is to arrive at some function.
The setup in which we work is therefore spaces of function. 
It is often the case that a classical solution to a given PDE can 
not be found. 
So we define a "weak derivative" using Lebesgue integrals, meaning 
the derivative exists in some strict mathematical sense. 

To use these integrals, we define $L_2(\Omega)$, which are all 
function $f$ over $\Omega$ such that 
\begin{equation*}
   \int_\Omega f^2 dx < \infty. 
\end{equation*}
We use this space to define a weak derivative.
\begin{defn}{Weak derivative}
    A function $f \in L_2(\Omega)$ has a weak derivative $g \in L_2(\Omega)$
    if
    \begin{equation*}
        \int_\Omega g(x)\phi(x) dx = (-1)^{|\alpha|}\int_\Omega 
        \partial ^{\alpha}\phi(x) f(x) dx
        \quad\quad \forall \phi \in C^\infty_0(\Omega).
    \end{equation*}
    We then write $g=\partial ^{\alpha}f$.
    \label{def:weak_derivative}
\end{defn}
We will use derivative and weak derivative interchangeably in this 
text.
If the derivative must be the normal, or strong, derivative, we will 
specify. 

We use Definition \ref*{def:weak_derivative} and $L_2(\Omega)$ spaces 
define the Hilbert space we are going to be working with. 
\begin{defn}{Hilbert Space}
   Let $m \geq 0$. Then $H^m(\Omega)$ is 
   \begin{equation*}
    H^m(\Omega) = \{  f \in L_2(\Omega) \mid \partial ^{\alpha}f \in 
    L_2(\Omega) \quad \forall |\alpha| \leq m  \}.
   \end{equation*}
   In other words, $H^m(\Omega)$ is the functions in $L_2(\Omega)$ 
   which contains weak derivatives up degree $m$.
\end{defn}

SNAK OM $H_0^m$ OG $C_0^{\infty}$. SIDE 29.