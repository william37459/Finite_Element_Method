\section{Hilbert Spaces}
When working with variational problems and FEM,
 the end goal is to arrive at some function which we call the solution.
The setup in which we work is therefore spaces of functions. 
It is often the case that a classical solution to a given PDE can 
not be found, 
so we define a `weak derivative' using Lebesgue integrals, meaning 
the derivative exists in some strict mathematical sense. 
To use these integrals, we define $L_2(\Omega)$, which are all 
function $f$ over $\Omega$ such that 
\begin{equation*}
   \int_\Omega f^2 dx < \infty. 
\end{equation*}
We use this space to define a weak derivative.
\begin{defn}{Weak Derivative}
   Let $\alpha$ be a $n$-dimensional multi-index, e.i. an $n$-dimensional vector, 
   with entries from the natural numbers.
    A function $f \in L_2(\Omega)$ has a weak derivative $g \in L_2(\Omega)$
    if
    \begin{equation*}
        \int_\Omega g(x)\phi(x) dx = {(-1)}^{|\alpha|}\int_\Omega 
        \partial ^{\alpha}\phi(x) f(x) dx
        \quad\quad \forall \phi \in C^\infty_0(\Omega).
    \end{equation*}
    We then write $g=\partial ^{\alpha}f$.\label{def:weak_derivative}
\end{defn}
We will use derivative and weak derivative interchangeably in this 
text.
If the derivative must be the classical, or strong, derivative, we will 
specify so explicitly. 
We use Definition~\ref*{def:weak_derivative} and $L_2(\Omega)$ to 
define the Hilbert space we are going to be working with. 
\begin{defn}{Sobolev Space}
   Let $m \geq 0$ be an integer. Then $H^m(\Omega)$ is 
   \begin{equation*}
    H^m(\Omega) = \{  f \in L_2(\Omega) \mid \partial ^{\alpha}f \in 
    L_2(\Omega) \quad \forall |\alpha| \leq m  \}.
   \end{equation*}
   In other words, $H^m(\Omega)$ is the set of functions in $L_2(\Omega)$ 
   which possess weak derivatives up to degree $m$.
\end{defn}
Since $H^m(\OO)$ is complete with respect to the $\| \cdot\|_m$ norm, 
$H^m(\OO)$ is also a Hilbert space, hence the $H$.
In~\cite{Brezis} one can find a proof for the following theorem, which 
we use for defining a specific subspace of $H^m(\OO)$.
\begin{thmx}{\quad}
   Let $\OO \subset \RR^n$ be an open set with piecewise smooth boundary, 
   and let $m \geq 0$. Then $C^\infty (\OO) \cap H^m(\OO)$ is dense in 
   $H^m(\OO)$.\label{thm:c_dense_in_h}
\end{thmx}
The reason we state Theorem~\ref{thm:c_dense_in_h} is that it implies a 
similar relationship for a smaller subset, namely $C^\infty_0(\OO)$, 
continuous differentiable functions with a compact support. So the following 
definition makes sense.
\begin{defn}{\quad}
  We define $H_0^m(\OO)$ be the completion of $C^\infty_0(\OO)$ with respect 
  to the Sobolev norm
 ($\| \cdot \|_m$).
\end{defn} 
Since $H_0^m(\OO)$ is a restriction on the functions in $H^m(\OO)$, 
we obviously know $H_0^m(\OO) \subset H^m(\OO)$ and, in the same 
sense, obviously $H^{n+1}(\OO) \subset H^n(\OO)$.
We now move on to discussing the norms on these spaces, 
which is the first step in figuring out how to `get closer' to 
the solution.