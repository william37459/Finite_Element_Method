\section{Boundary or Initial Value Problems}
When working with ODE's,
whether a problem explicitly states the boundary conditions 
or the initial values does not affect whether a problem can be solved.
When working with 
PDE's that is no longer the case. The different types 
of PDE's demand differently posed problems. 


----THE ABOVE: SKAL DET OVERHOVEDET MED?----


The theory of FEM springs from elliptic PDE's, so we will start 
describing these problems for elliptic PDE's.
We therefore assume that we can write a general linear PDE of second 
order with $n$ variables as 
\begin{equation*}
    Lu = f.
\end{equation*}
Since we are trying to solve the PDE over some space $\Omega$, we write
\begin{align}
    Lu &= f \quad \text{in } \Omega \label{eq:b_v_problem} \\
    u &= g \quad \text{on } \partial \Omega. \nonumber
\end{align}
A boundary condition on the form (\ref{eq:b_v_problem}) is called a 
Dirichlet condition, which is when $u$ is specified on the boundary. 
If $u=0$ we have homogeneous boundary conditions, which we usually assume 
to make things easier. If that is not the case, we can some times?????? 
transform the problem. To do this, we assume there exists a function 
$\tilde{u}$ such that $\tilde{u}=g$ on $\partial \Omega$. Then we define 
$w = u - \tilde{u}$ and $\tilde{f}=f-Lu$ and we get that
\begin{align}
    Lw &= \tilde{f} \quad \text{in } \Omega \label{eq:b_v_problem_homogeneous} \\
    w &= 0 \quad \text{on } \partial \Omega. \nonumber
\end{align}
To make these explicit demands for $u$ on $\partial \OO$ can 
be rather restrictive.
Instead we can make implicit restrictions, by constraining 
$\partial _i u$ on $\partial \OO$.
This will be dealth with later.