\section{Boundary or Initial Value Problems}
When working with ODE's,
whether a problem explicitly states the boundary conditions 
or the initial values does not affect whether a problem can be solved.
When working with 
PDE's that is no longer the case. The different types 
of PDE's demand differently posed problems. 


----THE ABOVE: SKAL DET OVERHOVEDET MED?----


The theory of FEM springs from elliptic PDE's, so we will start 
describing these problems for elliptic PDE's.
We therefore assume that we can write a general linear PDE of second 
order with $n$ variables as 
\begin{equation*}
    Lu = f.
\end{equation*}
Since we are trying to solve the PDE over some space $\Omega$, we write
\begin{align}
    Lu &= f \quad \text{in } \Omega \label{eq:b_v_problem} \\
    u &= g \quad \text{on } \partial \Omega. \nonumber
\end{align}
A boundary condition on the form (\ref{eq:b_v_problem}) is called a 
Dirichlet condition, which is when $u$ is specified on the boundary. 
If $u=0$ we have homogeneous boundary conditions.

A problem on the form (\ref{eq:b_v_problem}) can some times???????? 
be changed into a homogeneous form. Assume 
\begin{align}
    Lu &= f \quad \text{in } \Omega \label{eq:b_v_problem} \\
    u &= 0 \quad \text{on } \partial \Omega, \nonumber
\end{align}
which have what is called homogeneous boundary conditions. Of course, 
if $u=c$ on $\partial \Omega$ for some $c \in \RR$ it is 
rather simple process to convert it to the form (\ref{eq:b_v_problem}).