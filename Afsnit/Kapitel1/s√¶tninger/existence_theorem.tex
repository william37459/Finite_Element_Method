\begin{thmx}{Existence theorem}
    Let $L$ be a second order uniformly elliptic partial differential operator from~\eqref{def:pde_operator}. %TODO Hvorfor er det her?
     Then the homogeneous Dirichlet problem,
     \eqref{eq:b_v_problem_homogeneous}, always has a weak solution in $H_0^1(\OO)$. It is a minimum of the variational problem
    \begin{equation}
        \frac{1}{2} a(v,v) - {(f, v)}_0 \rightarrow \text{min}
    \end{equation}
    over $H_0^1(\OO)$.\label{thm:existence_hom_dirichlet}
\end{thmx}

\begin{bev}
    Let $c = \sup\{\left| a_{ik}(x) \right| \; | \; x \in \OO, 1 \leq i,k \leq n\}$. Then owing to the Cauchy-Schwarz inequality we get
    \begin{align}
    \begin{split}
        \left| \sum_{i,k} \int a_{ik} \partial_i u \partial_k v dx \right| &\leq c \sum_{i,k} \int  \left|  \partial_i u \partial_k v dx\right|\\
        &\leq c \sum_{i,k} {\left[ \int {\left( \partial_i u \right)}^2 dx \int {\left( \partial_k v \right)}^2 dx  \right]}^{1/2} \\
        &\leq C {\left| u \right|}_1 {\left| v \right|}_1 \label{eq:Cuv}
    \end{split}
    \end{align}
    where $C=cn^2$. 
    We also assume $C\geq \sup\{|a_0(x)| \; | \; x\in \Omega\}$ giving us
    \begin{equation}
        \left|\int a_0 u v dx \right| \leq C \int |u v| dx \leq C \|u\|_0 \|v\|_0.  
        \label{eq:estimate_a_0}   
    \end{equation}
    Then by~\eqref{eq:Cuv} and~\eqref{eq:estimate_a_0} we get continuity of $a$.
    Furthermore the uniform ellipticity implies the pointwise estimate %TODO Hvorfor gør den det?
    \[
        \sum_{i,k} a_{ik} \partial_i v \partial_k v \geq \alpha \sum_i {\left( \partial_i v \right)}^2,
    \]
    for $C^1(\OO)$ functions.
    As before, since $C^1(\OO)$ is dense in $H^1(\OO)$, all $u \in H^1(\OO)$ has a  %CHECK HER: Denne og næste ligning. Rimelig sikker på, at det er det her, han mener med density argument. Er ikke sikker på jeg har vendt rækkefølgen på H og C korrekt.
    representor in $C^1(\OO)$, and thus the previous statement can be generalized to $H^1(\OO)$.
    Integrating both sides and using $\text{a}_0 \geq 0$ leads to %TODO Er det fordi man fjerner noget i integralet?
    \begin{equation}
        a(v,v) \geq \alpha \sum_i \int_\OO {\left(\partial_i v\right)}^2 dx = \alpha |v|^2_1, \quad \forall v \in H^1(\OO).
        \label{eq:alpha_ellipticity}
    \end{equation}
    %TODO ref Friedrich equivalent %CHECK HER: Synes godt det kan slettes, der bliver referet til brødtekst
    We know from Friedrichs' Inequality $|\cdot|_1$ and $\| \cdot \|_1$ are equivalent norms on $H_0^1$,
     resulting in $a$ being an $H^1$-elliptic bilinear form on $H_0^1(\OO)$.
    From Theorem~\ref{thm:lax_milgram} there exists a unique weak solution which is also a solution of the variational problem.
\end{bev}