\begin{thmx}{Lax-Milgram Theorem}
Let $V$ be a closed, convex, non-empty set in a Hilbert space $H$, and let $a:H \times H \rightarrow \mathbb{R}$ be an elliptic bilinear form. Then for every $\ell\in H'$, the variational problem given by
\[
    J(v)=\frac{1}{2} a(v,v) - \ell(v) \longrightarrow \min    
\]
has a unique solution in $V$.\label{thm:lax_milgram}
\end{thmx}

\begin{bev}
    $J$ is bounded from below, since
    \begin{align*}
        J(v) &\geq \frac{1}{2} \alpha \|v\|^2 - \|\ell\| \, \|v\|\\
        &= \frac{1}{2\alpha} {(\alpha \|v\|-\|\ell\|)}^2 - \frac{\|\ell\|^2}{2\alpha} \\
        &\geq - \frac{\|\ell\|^2}{2\alpha}.
    \end{align*}
    We then let $c_1 = \inf\{J(v); v \in V\}$ and ${\{v_n \}}_{n=1}^\infty$ be a minimizing sequence. Then we get 
    \begin{align*}
        \alpha \|v_n-v_m\|^2 &\leq a(v_n-v_m,v_n-v_m) \\
        &= 2a(v_n,v_n) + 2a(v_m,v_m) - a (v_n+v_m,v_n+v_m) \\
        &= 4J(v_n) + 4J(v_m) - 8J(\frac{v_m+v_n}{2}) \\
        &\leq 4J(v_n) + 4J(v_m) - 8c_1. 
    \end{align*}
    By the convexity assumption on $V$, we get that $\frac{1}{2}(v_n + v_m) \in V$, and from that we get the last 
    inequality.
    We now get that ${\{v_n \}}_{n=1}^\infty$ is a Cauchy sequence in $H$ and $u = \lim_{n\rightarrow \infty}v_n$ exists given the fact that $J(v_n),J(v_m)\rightarrow c_1$ implies $\|v_n - v_m\| \rightarrow 0$ for $n,m\rightarrow \infty$.  

    Since $J$ is continuous,
    we see $J(u) = \lim_{n\rightarrow \infty} J(v_n) = \inf_{v\in V} J(v)$.
     Moreover, we also have that $u\in V$, because we assumed $V$ to be closed. 
    Lastly, we need to show that the solution is unique.
     To do that, we assume both $u_1$ and $u_2$ to be solutions of the variational problem. 
    Then we set up a minimizing sequence $u_1,u_2,u_1,u_2,\ldots$, which we know to be a Cauchy sequence from earlier.
     This implies $u_1 = u_2$.
\end{bev}