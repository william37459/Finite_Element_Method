\begin{thmx}{Trace Theorem}~\label{thm:trace}
    Let $\Omega$ be bounded, and suppose $\Omega$ has a piecewise smooth boundary. In addition, suppose $\Omega$ satisfies the cone condition. Then there exists a bounded linear mapping 
    \begin{equation*}
        \gamma : H^1(\Omega) \to L_2(\partial \Omega),
    \end{equation*}
    where $\|\gamma(v)\|_{0,\partial \OO} \leq c \|v\|_{1,\OO}$, and $\gamma (v) = v|_{\partial \OO}$, for all $v\in C^1(\OO)$.
\end{thmx}
\begin{bev}
    The presented proof will only be done in two dimensions, but can be generalised to higher dimensions.
    Suppose the boundary is piecewise smooth, and for the finitely many points where the boundary is not smooth the cone condition is satisfied. We can split the boundary into a finite number of smooth pieces $\partial \Omega_1, \partial \Omega_2, \dots, \partial \Omega_m$, where each $\partial \Omega_i$ after rotation of the coordinate system gives us;
    \begin{enumerate}
        \item For some function $\phi = \phi_i \in C^1[y_1, y_2]$, we have
            \[ \Gamma_i = \{ (x,y) \in \RR^2 \quad | \quad x = \phi(y), y_1 \leq y \leq y_2 \}. \]
        \item The domain $\Omega_i = \{ (x,y) \subset \RR^2 \quad | \quad \phi(y) < x <\phi(y) + r, y_1 < y < y_2 \}$ is contained in $\Omega$, where $r > 0$ is sufficiently small. 
    \end{enumerate}
    Firstly a function $v \in C^1(\bar{\Omega})$ and a point $(x,y) \in \partial \Omega$ can be written as
    \begin{equation}
            v(\phi(y), y) = v(\phi(y) + t, y) - \int_0^t \partial_1 v(\phi(y) + s, y) \, ds~\label{eq:trace_integral}
    \end{equation}
    where $0 \leq t \leq r$. By integrating~\eqref{eq:trace_integral} and switching the integration order we obtain 
    \begin{align*}
        \int_0^r v(\phi(y), y) dt &= \int_0^r \left(   v(\phi(y) + t, y) - \int_0^t \partial_1 v(\phi(y) + s, y) \, ds \right) dt \\
        r v(\phi(y), y) dt &= \int_0^r   v(\phi(y) + t, y) dt - \int_0^r \int_0^t \partial_1 v(\phi(y) + s, y) \, ds dt \\
        r v(\phi(y), y) dt &= \int_0^r   v(\phi(y) + t, y) dt - \int_0^r \int_s^r \partial_1 v(\phi(y) + s, y) \, dt ds \\
        r v(\phi(y), y) dt &= \int_0^r v(\phi(y) + t, y) dt - \int_0^r \partial_1 v(\phi(y) + t, y)(r-t) dt.
    \end{align*}
    By squaring the former equation, using a form of Young's inequality ${(a+b)}^2 \leq 2a^2 + 2b^2$, and applying the Cauchy-Schwarz inequality to the squares of the integrals results in
    \begin{equation}
        r^2 v^2(\phi(y), y) \leq 2 \int_0^r 1 dt \int_0^r v^2(\phi(y) + t, y) \, dt +  2\int_0^r t^2 \int_0^r | \partial_1  v(\phi(y) + t, y) |^2 dt.
    \end{equation}
    By subsituting $\int 1 dt = r$ and $\int t^2 dt = \frac{r^3}{3}$, dividing by $r^2$ and integrating over $y$, we obtain
    \begin{align*}
        \int_{y_1}^{y_2} v^2(\phi(y), y) dy &\leq  2r^{-1} \int_{\Omega_i} v^2 dx dy + \frac{2r}{3} \int_{\Omega_i} | \partial_1 v |^2 dx dy \\
        &\leq  2r^{-1} \int_{\Omega_i} v^2 dx dy + r \int_{\Omega_i} | \partial_1 v |^2 dx dy.
    \end{align*}
    On $\partial\Omega$ the arc length differential is given by $ds = \sqrt{1 + \phi'{(y)}^2} dy$. Thus, we have
    \begin{equation}
        \int_{\partial\Omega} v^2 ds \leq c_i \left[ 2r^{-1} \| v \|_0^2 + r | v |^2_1 \right],
    \end{equation}
    where $c_i = \text{max}\{\sqrt{1+{\phi'}^2} \,|\, y_1 \leq y \leq y_2 \}$. Setting $c=(r+2r^{-1})\sum_{i=1}^m c_i$, results in
    \begin{equation}
        \| v \|_{0,\partial\Omega} \leq c \| v \|_{1,\Omega}.
    \end{equation}
    Thus, the restriction $\gamma : H^1(\Omega) \cap C^1(\bar{\Omega}) \to L_2(\partial \Omega)$ is a bounded linear mapping on a dense subset of $H^1(\OO)$. Because of the completness of $L_2(\partial \Omega)$, it can be extended to all of $H^1(\Omega)$ without enlarging the bound.
\end{bev}