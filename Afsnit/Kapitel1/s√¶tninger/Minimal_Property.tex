\begin{thmx}{Minimal Property}
    Every classical solution of the boundary-value problem, given by
\begin{align}
    -\sum_{i,k} \partial_i (a_{ik}\partial_k u) + a_0 u &= f \quad \text{in } \Omega  \\
    u &= 0 \quad \text{on } \partial \Omega,
\end{align}
    is a solution to the variational problem, given by
    \[
        J(v)=\int_\Omega \left [\frac{1}{2}\sum_{i,k} a_{ik} \partial_i v\partial_k v + \frac{1}{2} a_0 v^2 -fv\right ]dx \longrightarrow \min,
    \]
    among all functions in $C^2(\Omega)\cap C^0(\bar{\Omega})$ with zero boundary values. 
\end{thmx}
\begin{bev}
    We start off by applying Green's formula
    \begin{equation}
    \label{eq:Greens_formula}
        \int_\Omega v\partial_i w dx = -\int_\Omega w \partial_i v dx + \int_{\partial \Omega} v w \vec{n}_i ds.
    \end{equation}
    Here we assume $v$ and $w$ to be $C^1(\OO)$ functions, and $\vec{n}_i$ is the $i$'th component of the outward-pointing normal $\vec{n}$.
    Now if we insert $w=a_{ik}\partial_k u$ in~\eqref{eq:Greens_formula}, we get
    \begin{equation}
    \label{eq:Greens_formula_inserted}
        \int_\Omega v\partial_i (a_{ik} \partial_k u) dx = -\int_\Omega a_{ik} \partial_i v v\partial_k u dx,
    \end{equation}
    given that $v=0$ on $\partial \Omega$.
    Now let 
    \begin{equation}
    \label{eq:a(u,v)}
        a(u,v) = \int_\Omega \left[\sum_{i,k} a_{ik} \partial_i u \partial_k v +a_0 uv \right]dx
    \end{equation}
    and
    \begin{equation}
    \label{eq:l(v)}
        \ell(v) = \int_\Omega fv dx.
    \end{equation}
    Then by summing~\eqref{eq:Greens_formula_inserted} over $i$ and $k$ we get that for every $v\in C^1(\Omega) \cap C(\bar{\Omega})$ with $v=0$ on $\partial \Omega$ we have
    \begin{align}
        a(u,v) - \ell(v) &= \int_\Omega v\left[ -\sum_{i,k} \partial_i (a_{ik} \partial_k u) + a_0 u - f \right] dx \label{eq:min_prop_variational_problem}\\
        &= \int_\Omega v [Lu - f] dx \nonumber \\
        &= 0, \nonumber
    \end{align}
    given that $Lu = f$. This property holds if $u$ is a classical solution.
     The minimal property is then implied by Theorem~\ref{thm:charac_theorem}
\end{bev}