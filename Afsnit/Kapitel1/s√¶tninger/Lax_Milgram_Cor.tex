The following corollary can be used to unify our approach.
\begin{kor}{\quad}
    Let $H$ be a complete Hilbert space, and let $V$ be a closed, linear subspace of $H$. The variational problem from Theorem~\ref{thm:lax_milgram} has a unique solution in $V$ given by\label{cor:lax_milgram}
   \begin{equation*}
    a(u,v) = \ell(v) \quad \forall v \in V.
   \end{equation*}
   \vspace{-8mm}
\end{kor}
\begin{bev}
    From the assumptions we know $H$ is closed and convex. Since $0 \in H$, $H$ is also non-empty. Then we can use Theorem~\ref{thm:charac_theorem} to find the form in the corollary.
\end{bev}
Theorem~\ref{thm:minimal_property}, Theorem~\ref{thm:lax_milgram} and Corollary~\ref{cor:lax_milgram} together shows that solving a Dirichlet problem is 
the same as finding the minimum of the corresponding variational problem. 
We can use this in our approach both theoretically and computationally.