
\begin{thmx}{Characterization Theorem}
    Let $V$ be a linear subspace, such that
        \(
            a: V \times V \rightarrow \mathbb{R}
        \) 
    is a strictly positive symmetric bilinear form
    %, then $a(v,v)>0$ for all $v \in V$, $v \neq 0$. In addition,
     and
        \(
            \ell: V \rightarrow \mathbb{R}
        \)
    is a linear functional. Then the quantity
        \[
            J(v)=\frac{1}{2} a(v,v) - \ell(v)
        \]
    obtains its minimum over $V$ at $u$ if and only if
        \begin{equation}
            a(u,v) = \ell(v) \quad \text{for all } v \in V.
        \label{eq:charac}
        \end{equation}
    There is at most one solution to~\eqref{eq:charac}.\label{thm:charac_theorem}
\end{thmx}

\begin{bev}
    For $u,v \in V$ and $t \in \mathbb{R}$ we have
    \begin{align}
        J(u+tv) &= \frac{1}{2} a(u+tv,u+tv) - \ell(u+tv) \nonumber \\
        &= \frac{1}{2} \left( a(u,u) + a(tv,tv) + 2a(u,tv) \right) - \left(  \ell(u) + \ell(tv)\right) \nonumber \\
        &= J(u) + t\left( a(u,v) - \ell(v) \right) + \frac{1}{2}t^2a(v,v). \label{eq:charac_proof_J(u+tv)}
    \end{align}
    If $u \in V$ satisfies~\eqref{eq:charac} and $t=1$, then from~\eqref{eq:charac_proof_J(u+tv)} we have
    \begin{alignat}{2}
        J(u+v) &= J(u) + \frac{1}{2}a(v,v) \quad &&\text{for all } v\in V  \nonumber \\
        &> J(u) \quad &&\text{for } v \neq 0,
    \end{alignat}
    since $a$ is positive.
    Thus $u$ is a unique minimal point. 
    To prove the opposite way, we assume that $J$ has a minimum at $u$.
    Then for every $v\in V$, the function $f(t)= J(u+tv)$ must fulfill the condition
    \begin{equation*}
        \frac{d}{dt}f(0) = 0,
    \end{equation*}
    since $J$ has a minimum at $u$, and as such any $t>0$ must increase the 
    value of $J$.
    Equation (\ref{eq:charac_proof_J(u+tv)}) is an equivalent statement to 
    the usual definition of differentiability, which gives us that 
     the derivative is $a(u,v) - \ell(v)$, and~\eqref{eq:charac} then follows.
\end{bev}