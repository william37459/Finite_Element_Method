
We will in this chapter apply the theory developed throughout the project to a specific example, which will be a modified 
version of some of the examples from~\cite{"fen-tutorial"}.
Here we will consider a problem with Dirichlet boundary conditions.
We wish to study convergence rate of the approximate solution, 
both in context of the size of cells and in degree used in the 
finite elements. 
The domain we will be working with is $\OO = \{ (x,y)\in \RR^2 \,|\, -1 \leq x,y \leq 1\}$, 
which is the square in $\RR^2$ with sidelength $2$, and center in origo.
We then define the solution 
\begin{equation}
    k(x,y) = \e^{x+y}\cos(x)\sin(y)+x  \quad\text{on }\Omega. \label{eq:app_dirichlet}
\end{equation}
We work backwards from the solution to be able to compare the numerical approximation 
with the real function. Using this $k$, we define the Dirichlet boundary problem to be
\begin{alignat}{2}
    Lu &= -\text{div }\nabla k \quad && \text{on } \Omega \\
    u &= k \quad && \text{on } \partial \Omega, \nonumber
\end{alignat}
Note in this example that, for simplicity, we have chosen the solution $k$ to be the same on the boundary as in the domain.
We will use the finite element method to approximate a solution for the problem. In this context we use the framework FEniCSx,
which is a computational tool for solving PDE's. 
FEniCSx consists of several different parts, which can be seen in~\cite{BarattaEtal2023},~\cite{ScroggsEtal2022}, and~\cite{BasixJoss}.