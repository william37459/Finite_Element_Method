To get a better overview of the error, 
we have made a graph of the error for different degrees of the polynomial 
and different sizes of $h$.
This makes it more managable to see how the the different degrees and mesh 
sizes affect the error. 
We see on Figure~\ref{fig:l2-fejl-plot} that in general the error 
decreases as the degree of the polynomial increases, 
although this is not the case for all mesh sizes.
We see that the error decreases as the mesh size decreases up 
to polynomials of degree $8$, 
which can be seen more accurately in Table~\ref{tab:convergence_l2}, 
but this trend changes after this point.
This trend should generally not be the case, 
but this could be due to the fact that the error is so small that the 
rounding errors become more significant.
The $H^1$ error shown on Figure~\ref{fig:h1-fejl-plot} generally 
has the same tendency as the $L_2$ error, 
but as mentioned earlier the error is generally higher.
However, the order of polynomial, where the tendency changes is 
different for the $H^1$ error, as it changes at degree $9$ instead of $8$, 
see Table~\ref{tab:convergence_H1}.
Furthermore, after the change in tendency, 
the error does not increase as drastically as for the $L_2$ error. 

\begin{figure}
    \begin{centering}
    \includegraphics[width=0.8\textwidth]{Afsnit/Application/figurer/l2-fejl-plot.jpeg}
    \caption{Plot of the $L_2$ error for the different degrees of the polynomial and different sizes of $h$.}
    \label{fig:l2-fejl-plot}
    \end{centering}
\end{figure}

\begin{figure}
    \begin{centering}
    \includegraphics[width=0.8\textwidth]{Afsnit/Application/figurer/h1-fejl_plot.png}
    \caption{Plot of the $H^1$ error for the different degrees of the polynomial and different sizes of $h$.}
    \label{fig:h1-fejl-plot}
    \end{centering}
\end{figure}