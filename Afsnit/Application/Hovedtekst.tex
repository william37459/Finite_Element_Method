
All references in the following section will be refering to the code listed in Appendix~\ref{app:Code}.

There are three big parts. The first is a solver, 
which solves a FEM-problem, using a certain number of cells, and a certain degree of polynomials. 
The second is a function which calculates the error in $L_2$,
which comes directly from~\cite{"fen-tutorial"}. 
The last part is two functions which loops over a different amount of cells and degree of polynomials, 
and outputs a table. 

We start on line 10 by defining our solution $u$ as the function from Equation~\eqref{eq:app_dirichlet},
after which we define two different interpolations of $u$ using the packages "numpy" and "ufl". 
We do this for convience, as numpy are better for numerical apporiximations, and ufl is 
easier to work with in FEniCSx.
Next, we need to define the mesh and the function space we will be working in. 
We do this on line 16-22, where $N$ defines the number of cells in each directions of our mesh, resulting in $N^2$ elements in total.
Then "degree" defines the degree of polynomials we use in each cell.
Next step in the process is to define the space of trial functions 
and the space of test functions, which we do on line 24-26.
In a different situation, these two spaces could have been different.

We then define the boundary condition, which is simply the function $k$, as mentioned earlier.
With the boundary condition established we can move on to defining the bilinear form $a$ and the linear form $L$.
With all this in place, we can now solve the problem using the PETSc package in FEniCSx. 
After obtaining the solution, we can look at the error, which can be computed using the $L_2$ norm.

%The convergence rate tells us how fast the error decreases as we increase the number of elements in our mesh.
%Ideally we would like the error to decrease as we decrease the mesh size. This corresponds to showing that the error $e=k_e-k_h$ is bounded by $\|e\|\leq Ch^r$, 
%with being the mesh size, $r$ the convergence rate and $C$ some constant independent of $h$.
The convergence rate tells us how fast the error decreases as we increase the number of elements in our mesh and degree of polynomials.
We will show this imperically by computing the error for different mesh sizes and degrees and comparing them. 
In Table~\ref{tab:convergence}, the error for five different mesh sizes, and 10 degrees of polynomials can be seen.

The table shows the error decreases as we decrease the cell size for all degrees of the polynomials, 
and smaller sizes of cells increases the convergence rate dramatically. 
For example, at $h=0.03125$, the error is less than $40$ already at degree $6$, and 
for $h=0.0625$ to reach the same error, we need a degree of $8$, while bigger cells do not 
even reach that size of error.
\begin{figure}[ht]
    \center
    % chktex-file 44
\begin{tabular}{|r|r|r|r|r|r|}
    \hline
    Degree &                    \multicolumn{5}{c|}{Size of $h$}                                   \\
    \hline
       &          0.25 &      0.125       &           0.0625 &          0.03125 &         0.015625 \\
    \hline
    1  & 4.67454e+08   &      2.25264e+08 &      7.91207e+07 &      2.20585e+07 &      5.73871e+06 \\
    \hline
    2  & 2.25595e+08   &      7.90539e+07 &      1.13243e+07 &      1.49544e+06 &           190463 \\
    \hline
    3  & 1.26809e+08   &      1.79289e+07 &      1.32999e+06 &           113656 &           7390.8 \\
    \hline
    4  & 5.12053e+07   &      2.95071e+06 &           300970 &          12221.9 &          402.822 \\
    \hline
    5  & 1.60739e+07   &      1.3077e+06  &          49422.9 &          847.042 &          13.3082 \\
    \hline
    6  & 4.90332e+06   &      416519      &          4436.92 &          34.2407 &         0.272078 \\
    \hline
    7  & 2.93235e+06   &      71557.2     &          234.893 &          1.46575 &       0.00644169 \\
    \hline
    8  & 1.41838e+06   &      4687.31     &          38.2869 &         0.105716 &      0.000220462 \\
    \hline
    9  & 432549        &      1346.04     &          4.40185 &        0.0047685 &       1.4813e-05 \\
    \hline
    10 & 70746.3       &      378.437     &         0.270403 &       0.00012822 &      4.77788e-05 \\
    \hline
\end{tabular}
    \caption{Table of convergence rate of errors}\label{tab:convergence}
\end{figure}