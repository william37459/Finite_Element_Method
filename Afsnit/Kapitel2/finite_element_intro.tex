In this chapter examine ways to find the solutions described in the previous chapter.
The idea is essentially to partition the domain $\OO$ into a finite amount 
of subsets, preferably with certain qualities, which we will discuss later. 

Using these subsets, we can define a subspace of $H^m(\OO)$ or $H^m_0(\OO)$ 
with a finite dimension, called $S_h$. The variational problem, $J(v)$, can then be 
solved over this space, since an infinite number of dimensions can make 
computation difficult, or impossible.

There will be $2$ parts; in one part we 
will examine how the solution of $J$ behaves in $S_h$ versus the full space, 
and in the other we will discuss partitioning $\OO$, using a mesh.
The terminology in the literature is as follows: An \emph{element} or \emph{cell} is a 
subset of $\OO$, meaning a geometric object, and a \emph{finite element} is a %TODO Ikke korrekt, se def. 5.8 - slet
function from $S_h$. When the context makes the meaning clear, we might deviate from 
this convention.
We will start by limiting our discussion to a polygonial $\OO \in \RR^2$, which 
can be partitioned into triangles or quadrilaterals.