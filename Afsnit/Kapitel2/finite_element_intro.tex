In this chapter we examine ways to numerically approximate the solutions described in the previous chapter.
The idea is essentially to partition the domain $\OO$ into a finite amount 
of subsets, preferably with certain qualities, which we will discuss later. 

Using these subsets, we can define a subspace of $H^m(\OO)$ or $H^m_0(\OO)$ 
with a finite dimension, called $S_h$. 
This could be a space of piecewise polynomials. 
The variational problem can then be 
solved over this space, since an infinite number of dimensions can make 
computation difficult, or impossible.

There will be $2$ parts; in one we 
will examine how the minimizers of $J$ behaves in $S_h$ versus the full space, 
and in the other we will discuss partitioning $\OO$, using a mesh.
An \emph{element} or \emph{cell} is a 
subset of $\OO$, meaning a geometric object, 
and a \emph{finite element} is a 
triple consisting of; a subset of $\OO$, a space of functions, and a set of 
linearly independent functionals on these functions.
When the context makes the meaning clear, we might deviate from 
this convention.
We will start by limiting our discussion to a polygonial $\OO \subset \RR^2$, which 
can be partitioned into triangles or quadrilaterals.