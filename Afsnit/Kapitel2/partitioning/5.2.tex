
\begin{thmx}{\quad\label{thm:h-and-c_connection}}
    Let $k\geq1$ and suppose $\Omega$ is bounded. Then a piecewise infinitely differentiable function $v:\bar{\Omega}\rightarrow \mathbb{R}$
    belongs to $H^k(\Omega)$ if and only if $v\in C^{k-1}(\bar{\Omega})$.
\end{thmx}

\begin{bev}
    For simplicity we restrict ourselves to domains in $\mathbb{R}^2$.
    Start by assuming $k=1$.
    Assuming $v\in C^0(\bar\Omega)$ we wish to prove $v\in H^1(\OO)$. Let $\mathcal{T}={\{T_j\}}^M_{j=1}$ be a partition of $\Omega$.
    Define the piecewise functions $w_i:\Omega\rightarrow \mathbb{R}$ for $i=1,2$ as $w_i(x)=\partial_i v(x)$ for $x\in\Omega$,
     where the function can take either of the two limiting values on the edges of $T_j$. %FIXIT: Måske uddyb det her lidt mere (Anton svar)
    \\
    Let $\phi\in C^{\infty}_0 (\Omega)$
   %  and using Green's identity on every element $T_j$ we get
   and we get
    \begin{align}
    \begin{split}
    \int_\Omega \phi w_i dxdy &= \sum_j\int_{T_j} \phi \partial_i v dx dy \\
        &= \sum_j \left( -\int_{T_j} \partial_i \phi v dxdy + \int_{\partial T_j} \phi v \mathbf{n}_i ds\right).
    \end{split}
    \end{align}
    The first equality follows from linearity of integrals and qualities of a partition, and the second equality is Greens identity.
    Since $v$ is continuous, 
    the integrals over the interior edges cancel out,
    as the normal vectors point in opposite directions.
     The function $\phi$ has compact support, 
     meaning it is zero on the boundary and outside $\OO$. 
     The integrals on the boundary therefore vanish and we are left with
    \begin{equation}
        \int_\Omega \phi w_i dxdy = -\int_\Omega \partial_i \phi v dxdy.
    \end{equation}
    This, by Definition~\ref{def:weak_derivative}, implies that $w_i$ is the first order weak derivative of $v$.
    As $v\in C^0(\bar\Omega)$ we get $v \in L_2(\OO)$. These facts together results in $v\in H^1(\OO)$ by definition.

    We now examine the other implication. 
    Assuming $v\in H^1(\OO)$ and piecewise differentiable, we wish to show $v \in C^0(\bar{\OO})$. 
    First we examine $v$ in the neighborhood of an edge of an element, and rotate the edge so it lies on the $y$-axis. 
    On the edge we define the interval $[y_1-\delta, y_2+\delta]$ for $y_1$ and $y_2$ on the edge, and $y_1<y_2$, and 
    $\delta > 0$.
    Define the auxiliary function
    \begin{equation}
        \psi (x) = \int_{y_1}^{y_2} v(x,y) dy.
    \end{equation}
    The auxiliary function has the following properties
    \begin{equation}
        \psi ' =\int_{y_1}^{y_2} \partial_1 v dy, \quad \psi(x_2) - \psi(x_1) =\int_{x_1}^{x_2} \psi ' dx.
    \end{equation}
    Suppose $v \in C^{\infty}(\OO)$. From the Cauchy-Scwarz inequality we get
    \begin{align}
        |\psi(x_2) -\psi(x_1)|^2 &= \left| \int_{x_1}^{x_2} \int_{y_1}^{y_2} \partial_1 v dx dy \right|^2\\
        &\leq \left| \int_{x_1}^{x_2} \int_{y_1}^{y_2} 1 dx dy \right|^2 \cdot | v |_{1, \OO}^2\\
        &\leq |x_2 - x_1|^2 \cdot |y_2 - y_1|^2 \cdot | v |_{1, \OO}^2,
    \end{align}
    which is Lipschitz continuity.
    Reusing the same density argument, since $C^\infty(\OO)$ is dense in $H^1(\OO)$, the previous inequality 
    also holds for $v\in H^1(\OO)$.
    The function $\psi$ is thus continuous for the entire domain, and at $x=0$. 
    Since $y_1$ and $y_2$ was arbitrarily chosen, it must be true for the entire edge,
    so $v$ is continuous on $\OO$. 
    The boundary of $\OO$ will be included in the edges of the elements,
    and therefore we get $v\in C^0(\bar{\OO})$.

    For $k>1$, assume the theorem holds for $k-1$. We can then, by assumption, differentiate 
    $k-1$ times, where the bi-implication holds after each differentiation. 
    After differentiating $k-1$ times, we arrive at the case we have just proven. 
    The theorem is thus true for $k>1$.
\end{bev}
In Theorem~\ref{thm:h-and-c_connection} $\OO$ could be the entire domain; 
however it could also just be a cell. This gives us some local requirements. 
If we want a solution which is locally $C^2$ (not on a cell edge), every function in the 
finite elements must be $H^3$.