\begin{lem}{Cea's Lemma\label{lem:cea}}
    Let the bilinear form $a$ be $V$-elliptic with $H_0^m(\Omega)\subset V \subset H^m(\Omega)$. In addition, let $u$ and $u_h$ be solutions to the variational problem in $V$ and $S_h\subset V$, respectively. Then
    \begin{equation}
        \label{eq:cea}
        \|u-u_h\|_m\leq \frac{C}{\alpha}\inf_{v_h\in S_h} \|u-v_h\|_m.
    \end{equation}
    
\end{lem}
\begin{bev}
    The way we define $u$ and $u_h$ yields
    \begin{align}
    \begin{split}
        a(u,v)&= \ell(v) \quad \forall v\in V,\\
        a(u_h,v)&=\ell(v) \quad \forall v\in S_h.
    \end{split}
    \end{align}
    We assumed $S_h\subset V$, and we therefore by subtraction get
    \begin{equation}
        a(u,v)-a(u_h,v)=0 \quad \forall v\in S_h.
    \end{equation}
    which by linearity of $a$ yields
    \begin{equation}
        \label{eq:cea_proof}
        a(u-u_h,v)=0 \quad \forall v\in S_h.
    \end{equation}
    Now let $v_h\in S_h$ and $v=v_h-u_h$. Then combining this with~\eqref{eq:cea_proof}, we get  

    \begin{equation}
        a(u-u_h,v_h-u_h)=0 \quad \forall v_h\in S_h,
    \end{equation}
    and
    \begin{align*}
        \alpha\|u-u_h\|_m^2&\leq a(u-u_h,u-u_h)\\
        &=a(u-u_h,u-v_h)+a(u-u_h,v_h-u_h)\\ 
        &\leq C\|u-u_h\|_m\|u-v_h\|_m.
    \end{align*}
    Rearranging the inequality yields 
    \begin{equation}
        \|u-u_h\|_m\leq \frac{C}{\alpha} \|u-v_h\|_m,
    \end{equation} 
    and therefore
    \begin{equation}
        \|u-u_h\|_m\leq \frac{C}{\alpha}\inf_{v_h\in s_h} \|u-v_h\|_m.
    \end{equation}
\end{bev}
Lemma~\ref{lem:cea} gives us an understanding of possible errors; the next 
theorem shows a connection between our Hilbert spaces and continious 
functions.