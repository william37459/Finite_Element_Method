The idea of $S_h$ is to solve the same variational problem as in the global case, 
but in a smaller space; so the problem becomes
\begin{equation*}
    J(v) = \frac{1}{2}a(v,v) - \ell(v) \to \underset{S_h}{\min}.
\end{equation*}
The only difference is in the space we look for a solution, namely $S_h$. From 
earlier theory, we know that the solution is given by an $u_h$ which solves
\begin{equation*}
    a(u_h,v) = \ell(v) \quad \forall v \in S_h.
\end{equation*}
From this point on, for a solution $u$ to some variational problem, $u_h$ will be the solution in $S_h$.
Since we assumed $S_h$ to be finite dimensional, we can let $ \{ \psi_1, \psi_2, \ldots, \psi_N \}$ 
be a basis for $S_h$. Since $\ell$ is a linear functional and $a$ is bilinear, 
$u_h$ can be found using the following weak formulation:
\begin{equation*}
    a(u_h, \psi_i) = \ell(\psi_i) \quad i = 1, 2, \ldots, N.
\end{equation*}
As $u_h \in S_h$, $u_h$ can be found to be a linear combination of the basis-fuctions, with 
coefficients $z_i$, $i=1, 2, \ldots, N$. As $a$ is bilinear, we get the system of 
equations
\begin{equation*}
    \sum_{k=1}^N z_k a(\psi_k,\psi_i) = \ell(\psi_i) \quad i = 1,2,\ldots,N.
\end{equation*}
Letting $a(\psi_i,\psi_j)$ be the $j,i$-entries in a matrix $A$, $z_i$ and $\ell(\psi_i)$ 
be the entries the vectors $z$ and $b$ respectively, this can be written in matrix form 
\begin{equation}
    Az = b. \label{eq:matrix_equation}
\end{equation}
When using FEM, Equation~\eqref{eq:matrix_equation} is what we use to find the solution; 
the other big part of FEM is constructing $A$, which is dependent on the partitioning of $\OO$, 
as this partitioning decides what $S_h$ and the basisvectors are.

It can be shown that the matrix $A$ is symmetric and positive definite, when $a$ is an $H^m$-elliptic bilinear form, which we do now. 
The symmetry of $A$ follows directly from the symmetry of $a$, as $a$ is $H^m$-elliptic.
To show that $A$ is positive definite, we need to show that $x^T Ax > 0$ for all $x \neq 0$. Thus
\begin{align*}
    x^T Ax &= \sum_{i,j} x_i A_{ik}x_k \\
    &= a\left(\sum_{k} x_k\psi_k,\sum_{i} x_i\psi_i\right) \\
    &= a(u_h,u_h) \geq \alpha \|u_h\|^2 > 0.
\end{align*}

From now on, we will assume for simplicity that $V\subset H^m(\Omega)$ and $a$ is $V$-elliptic. %TODO Uddyb hvad V er

