\section{Partitioning}
Partitioning some space simply means splitting it into a bunch of subspaces. 
There are different ways to do that and different desirable qualities. 
The important one will be covered later.

By partitioning $\OO$, we also create a new space of functions over $\OO$. 
By splicing the function over each cell, we can create all kinds of 
discontinuous functions
; this, however, does not help us reach our goal, 
as many of these functions are not necessarily differentiable. 
For each cell, we create certain requirements on the edges for the functions 
over this cell. The functions which obey these requirements make up $S_h$.
The idea of $S_h$ is to solve the same variational problem as in the global case, 
but in a smaller space; so the problem becomes
\begin{equation*}
    J(v) = \frac{1}{2}a(v,v) - \ell(v) \to \underset{S_h}{\min}.
\end{equation*}
The only difference is in the space we look for a solution, namely $S_h$. From 
earlier theory, we know that the solution is given by an $u_h$ which solves
\begin{equation*}
    a(u_h,v) = \ell(v) \quad \forall v \in S_h.
\end{equation*}
Since we assumed $S_h$ to be finite dimensional, we can let $ \{ \psi_1, \psi_2, \ldots, \psi_N \}$ 
be a basis for $S_h$. Since $\ell$ is a linear functional and $a$ is bilinear, 
$u_h$ can be found using the following weak formulation:
\begin{equation*}
    a(u_h, \psi_i) = \ell(\psi_i) \quad i = 1, 2, \ldots, N.
\end{equation*}
As $u_h \in S_h$, $u_h$ can be found to be a linear combination of the basis-fuctions, with 
coefficients $z_i$, $i=1, 2, \ldots, N$. As $a$ is bilinear, we get the system of 
equations
\begin{equation*}
    \sum_{k=1}^N z_k a(\psi_k,\psi_i) = \ell(\psi_i) \quad i = 1,2,\ldots,N.
\end{equation*}
Letting $a(\psi_i,\psi_j)$ be the $j,i$-entries in a matrix $A$, $z_i$ and $\ell(\psi_i)$ 
be the entries the vectors $z$ and $b$ respectively, this can be written in matrix form 
\begin{equation}
    Az = b. \label{eq:matrix_equation}
\end{equation}
When using FEM, Equation~\eqref{eq:matrix_equation} is what we use to find the solution; 
the other big part of FEM is constructing $A$, which is dependent on the partitioning of $\OO$, 
as this partitioning decides what $S_h$ and the basisvectors are.

It can be shown that the matrix $A$ is symmetric and positive definite, when $a$ is an $H^m$-elliptic bilinear form, which we do now. 
The symmetry of $A$ follows directly from the symmetry of $a$, as $a$ is $H^m$-elliptic.
To show that $A$ is positive definite, we need to show that $x^T Ax > 0$ for all $x \neq 0$. Thus
\begin{align*}
    x^T Ax &= \sum_{i,j} x_i A_{ik}x_k \\
    &= a\left(\sum_{k} x_k\psi_k,\sum_{i} x_i\psi_i\right) \\
    &= a(u_h,u_h) \geq \alpha \|u_h\|^2 > 0.
\end{align*}

From now on, we will assume for simplicity that $V\subset H^m(\Omega)$ and $a$ is $V$-elliptic. %TODO Uddyb hvad V er



\begin{defn}{Admissible Partition}
   A partition of $\OO$ into triangles or quadrilaterals, using 
   $\mathcal{T} = \{T_1, T_2, \ldots, T_M \}$, is called admissible if:~\label{def:admissible_partition}
   \begin{enumerate}
    \item $ \bar{\OO} = \cup_{i=1}^M T_i$
    \item If $T_i \cap T_j$ is exactly one point, it is a common vertex of 
    $T_i$ and $T_j$
    \item If, for $i\neq j$, $T_i \cap T_j$ is more than one point, it is a 
    common edge of $T_i$ and $T_j$.
   \end{enumerate}
\end{defn}
We further describe these partitions. If every element in $\mathcal{T}$ has points 
which have at most a distance of $2h$, or equivalently a diameter of $2h$, 
we write $\mathcal{T}_h$. %TODO Family of partitions?
\begin{lem}{Cea's Lemma}
    Let the bilinear form $a$ be $V$-elliptic with $H_0^m(\Omega)\subset V \subset H^m(\Omega)$. In addition, let $u$ and $u_h$ be solutions to the variational problem in $V$ and $S_h\subset V$, respectively. Then
    \begin{equation}
        \label{eq:cea}
        \|u-u_h\|_m\leq \frac{C}{\alpha}\inf_{v_h\in S_h} \|u-v_h\|_m.
    \end{equation}
\end{lem}
\begin{bev}
    The way we define $u$ and $u_h$ yields
    \begin{align}
    \begin{split}
        a(u,v)&= \ell(v) \quad \forall v\in V,\\
        a(u_h,v)&=\ell(v) \quad \forall v\in S_h.
    \end{split}
    \end{align}
    We assumed $S_h\subset V$, and we therefore by subtraction get
    \begin{equation}
        a(u,v)-a(u_h,v)=0 \quad \forall v\in S_h.
    \end{equation}
    which by linearity of $a$ yields
    \begin{equation}
        \label{eq:cea_proof}
        a(u-u_h,v)=0 \quad \forall v\in S_h.
    \end{equation}
    Now let $v_h\in S_h$ and $v=v_h-u_h$. Then combining this with~\eqref{eq:cea_proof}, we get  

    \begin{equation}
        a(u-u_h,v_h-u_h)=0 \quad \forall v_h\in S_h,
    \end{equation}
    and
    \begin{align*}
        \alpha\|u-u_h\|_m^2&\leq a(u-u_h,u-u_h)\\
        &=a(u-u_h,u-v_h)+a(u-u_h,v_h-u_h)\\ 
        &\leq C\|u-u_h\|_m\|u-v_h\|_m.
    \end{align*}
    Rearranging the inequality yields 
    \begin{equation}
        \|u-u_h\|_m\leq \frac{C}{\alpha} \|u-v_h\|_m,
    \end{equation} 
    and therefore
    \begin{equation}
        \|u-u_h\|_m\leq \frac{C}{\alpha}\inf_{v_h\in s_h} \|u-v_h\|_m.
    \end{equation}
\end{bev}

\begin{thmx}{\quad\label{thm:h-and-c_connection}}
    Let $k\geq1$ and suppose $\Omega$ is bounded. Then a piecewise infinitely differentiable function $v:\bar{\Omega}\rightarrow \mathbb{R}$
    belongs to $H^k(\Omega)$ if and only if $v\in C^{k-1}(\bar{\Omega})$.
\end{thmx}

\begin{bev}
    For simplicity we restrict ourselves to domains in $\mathbb{R}^2$.
    Start by assuming $k=1$.
    Assuming $v\in C^0(\bar\Omega)$ we wish to prove $v\in H^1(\OO)$. Let $\mathcal{T}={\{T_j\}}^M_{j=1}$ be a partition of $\Omega$.
    Define the piecewise functions $w_i:\Omega\rightarrow \mathbb{R}$ for $i=1,2$ as $w_i(x)=\partial_i v(x)$ for $x\in\Omega$,
     where the function can take either of the two limiting values on the edges of $T_j$. %FIXIT: Måske uddyb det her lidt mere (Anton svar)
    \\
    Let $\phi\in C^{\infty}_0 (\Omega)$
   %  and using Green's identity on every element $T_j$ we get
   and we get
    \begin{align}
    \begin{split}
    \int_\Omega \phi w_i dxdy &= \sum_j\int_{T_j} \phi \partial_i v dx dy \\
        &= \sum_j \left( -\int_{T_j} \partial_i \phi v dxdy + \int_{\partial T_j} \phi v \mathbf{n}_i ds\right).
    \end{split}
    \end{align}
    The first equality follows from linearity of integrals and qualities of a partition, and the second equality is Greens identity.
    Since $v$ is continious, 
    the integrals over the interior edges cancel out,
    as the normal vectors point in opposite directions.
     The function $\phi$ has compact support, 
     meaning it is zero on the boundary and outside $\OO$. 
     The integrals on the boundary therefore vanish and we are left with
    \begin{equation}
        \int_\Omega \phi w_i dxdy = -\int_\Omega \partial_i \phi v dxdy.
    \end{equation}
    Which by Definition~\ref{def:weak_derivative} implies that $w_i$ is the first order weak derivative of $v$.
    As $v\in C^0(\bar\Omega)$ we get $v \in L_2(\OO)$. These facts together results in $v\in H^1(\OO)$ by definition.

    We now examine the other implication. 
    Assuming $v\in H^1(\OO)$ we wish to show $v \in C^0(\bar{\OO})$. 
    First we examine $v$ in the neighborhood of an edge of an element, and rotate the edge so it lies on the $y$-axis. 
    On the edge we define the interval $[y_1-\delta, y_2+\delta]$ for $y_1$ and $y_2$ on the edge, and $y_1<y_2$, and 
    $\delta > 0$.
    Define the auxiliary function
    \begin{equation}
        \psi (x) = \int_{y_1}^{y_2} v(x,y) dy.
    \end{equation}
    The auxiliary function has the following properties
    \begin{equation}
        \psi ' =\int_{y_1}^{y_2} \partial_1 v dy, \quad \psi(x_2) - \psi(x_1) =\int_{x_1}^{x_2} \psi ' dx.
    \end{equation}
    Suppose $v \in C^{\infty}(\OO)$. From the Cauchy-Scwarz inequality we get
    \begin{align}
        |\psi(x_2) -\psi(x_1)|^2 &= \left| \int_{x_1}^{x_2} \int_{y_1}^{y_2} \partial_1 v dx dy \right|^2\\
        &\leq \left| \int_{x_1}^{x_2} \int_{y_1}^{y_2} 1 dx dy \right| \cdot | v |_{1, \OO}^2\\
        &\leq |x_2 - x_1| \cdot |y_2 - y_1| \cdot | v |_{1, \OO}^2,
    \end{align}
    which is Lipschitz continuity.
    Reusing the same density argument, since $C^\infty(\OO)$ is dense in $H^1(\OO)$, the previous inequality 
    also holds for $v\in H^1(\OO)$.
    The function $\psi$ is thus continious for the entire domain, and at $x=0$. 
    Since $y_1$ and $y_2$ was arbitrarily chosen, it must be true for the entire edge,
    so $v$ is continious on $\OO$. 
    The boundary of $\OO$ will be included in the edges of the elements,
    and therefore we get $v\in C^0(\bar{\OO})$.

    For $k>1$, assume the theorem holds for $k-1$. We can then, by assumption, differentiate 
    $k-1$ times, where the bi-implication holds after each differentiation. 
    After differentiating $k-1$ times, we arrive at the case we have just proven. 
    The theorem is thus true for $k>1$.
\end{bev}
In Theorem~\ref{thm:h-and-c_connection} $\OO$ could be the entire domain; 
however it could also just be a cell. This gives us some local requirements. 
If we want a solution which is locally $C^2$ (not on a cell edge), every function in the 
finite elements must be $H^3$.
We now move from discussing $\OO$, to what a finite element is,
and how to construct them.