\begin{defn}{Finite Element}
    A finite element is a triple $(T, \Pi,\Sigma)$ which has the following properties:
    \label{def:finite_element}
    \begin{enumerate}
        \item $T\subset \RR^d$ is a polyhedron
        \item $\Pi \subset C(T)$ with finite dimension $s$
        \item $\Sigma$ is a set of $s$ linearly independent functionals on $\Pi$. 
        Every $p\in \Pi$ is uniquely defined by the values of the $s$ functionals in $\Sigma$.
    \end{enumerate}
\end{defn}
The vernacular can be extended as follows:
\begin{itemize}
    \item The parts of $\partial T$ which lies in different hyperplanes are called faces
    \item A set of functions in $\Pi$ which form a basis are called shape functions 
    \item $s$ is the number of local degrees of freedom or local dimension
\end{itemize}
If taken head on, whenever we partition $\OO$, we would have to examine 
each finite element individually, which would make this method completely 
untenable. What we do instead is examine a single finite element, which 
can represent all the other finite elements.
In the following text we will use the set of polynomials of two variables 
with degree $t$, which are defined as 
\begin{equation*}
\mathcal{P}_t = \{ f(x,y) = \sum_{\substack{i+k \leq t\\ 0\leq i,k}} c_{ik}x^i y^k \}.
\end{equation*}
If a finite element $(T, \Pi_i,\Sigma_i)$ fulfills 
    $\mathcal{P}_t \subset \Pi_i$,
we call it a finite element with complete polynomials (of degree $t$, if 
it is not clear).