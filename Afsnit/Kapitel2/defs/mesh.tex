
\begin{defn}{Admissible Partition}
   A partition of $\OO$ into triangles or quadrilaterals, using 
   $\mathcal{T} = \{T_1, T_2, \ldots, T_M \}$, is called admissible if:~\label{def:admissible_partition}
   \begin{enumerate}
    \item $ \bar{\OO} = \cup_{i=1}^M T_i$
    \item If $T_i \cap T_j$ is exactly one point, it is a common vertex of 
    $T_i$ and $T_j$
    \item If, for $i\neq j$, $T_i \cap T_j$ is more than one point, it is a 
    common edge of $T_i$ and $T_j$.
   \end{enumerate}
\end{defn}
We further describe these partitions. If every element in $\mathcal{T}$ has points 
which have at most a distance of $2h$, or equivalently a diameter of $2h$, 
we write $\mathcal{T}_h$. We also call $\mathcal{T}$ a \emph{family of partitions}. 
This is common vernacular to all the elements used to partition $\OO$. 

Intuitively, an admissible partition of $\OO$ is nice; we can partition $\OO$ 
into triangles (or quadrilaterals) without losing parts of $\OO$, and no 
elements contain random points scattered around $\OO$ without being connected 
to the element itself.

Next we examine a way to dominate error of using a solution in $S_h$ instead 
of $V$.