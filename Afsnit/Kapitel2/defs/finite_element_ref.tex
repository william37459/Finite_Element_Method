\iffalse \begin{defn}{Affine Family}
    A family of finite element spaces $S_h$ for partitions $\mathcal{T}_h$ of 
    $\OO\subset \RR^d$ is called an affine family if there exists a finite reference
    element $(T_{\text{ref}}, \Pi_{\text{ref}},\Sigma)$, such that, beside the 
    usual properties of a finite element, there exists some affine mappings 
    $F_j:T_{\text{ref}}\to T_j$ such that
    \begin{equation*}
        \forall v \in S_h \,\,\land \,\, \forall x \in T_j:\,\, v(x) = p(F^{-1}_j(x)) \quad 
        p \in \Pi_{\text{ref}}.
    \end{equation*}
    Meaning, when restricting $v\in S_h$ to some $T_j$, we can fully describe 
    $v$ using some affine mapping and an element in $\Pi_{\text{ref}}$.
\end{defn}
\fi
\begin{defn}{Affine Family} 
    Let $(T_{\text{ref}}, \Pi_{\text{ref}},\Sigma_{\text{ref}})$ be a finite element 
    with $T_{\text{ref}}\in \mathcal{T}$ for som admissable partition of 
    $\OO$, and 
    $F$ an affine linear transformation. 
    Assume that for $T_i\in\mathcal{T}$ the following 
    is true for the corresponding finite element $(T_i, \Pi_i,\Sigma_i)$:
    \begin{itemize}
        \item $F(T_{\text{ref}}) = T_i$
        \item $\{ f\circ F \,|\,  f \in \Pi_i \} =\Pi_{\text{ref}}  $
        \item $\{ s(f\circ F) \,|\, f \in \Pi_i, s \in \Sigma_{\text{ref}} \} = \Sigma_i$
    \end{itemize}
    If the previous equalities are true for all $T_i\in \mathcal{T}$ we call 
    $(T_{\text{ref}}, \Pi_{\text{ref}},\Sigma_{\text{ref}})$ the finite reference 
    element.
\end{defn}
The previous definition might be a bit dense, and so we will expand a bit 
further here. The first equality in the definition requires that the 
different cells are "similar enough". The second equality makes sure 
that the functions in the $i$'th finite element corresponds 
to using the same  
functions on the transformed reference cell - and that these are the 
same functions used in the reference finite element.
The final equality ensures that the points in the reference cell that 
uniquely identify the functions in the reference finite elements, 
remain in the transformed cell.