\iffalse \begin{defn}{Affine Family}
    A family of finite element spaces $S_h$ for partitions $\mathcal{T}_h$ of 
    $\OO\subset \RR^d$ is called an affine family if there exists a finite reference
    element $(T_{\text{ref}}, \Pi_{\text{ref}},\Sigma)$, such that, beside the 
    usual properties of a finite element, there exists some affine mappings 
    $F_j:T_{\text{ref}}\to T_j$ such that
    \begin{equation*}
        \forall v \in S_h \,\,\land \,\, \forall x \in T_j:\,\, v(x) = p(F^{-1}_j(x)) \quad 
        p \in \Pi_{\text{ref}}.
    \end{equation*}
    Meaning, when restricting $v\in S_h$ to some $T_j$, we can fully describe 
    $v$ using some affine mapping and an element in $\Pi_{\text{ref}}$.
\end{defn}
\fi
\begin{defn}{Affine Family} %TODO Færdiggør Engberg
    Let $(T_{\text{ref}}, \Pi_{\text{ref}},\Sigma_{\text{ref}})$ be a finite element 
    with $T_{\text{ref}}\in \mathcal{T}$ for som admissable partition of 
    $\OO$, and 
    $F$ an affine linear transformation. 
    Assume that for $T_i\in\mathcal{T}$ the following 
    is true for the corresponding finite element $(T_i, \Pi_i,\Sigma_i)$:
    \begin{itemize}
        \item $F(T_{\text{ref}}) = T_i$
        \item $\{ f\circ F \,|\,  f \in \Pi_i \} =\Pi_{\text{ref}}  $
        \item $\{ s(f\circ F) \,|\, f \in \Pi_i, s \in \Sigma_{\text{ref}} \} = \Sigma_i$
    \end{itemize}
    If the previous equalities are true for all $T_i\in \mathcal{T}$ we call 
    $(T_{\text{ref}}, \Pi_{\text{ref}},\Sigma_{\text{ref}})$ the finite reference 
    element.
\end{defn}