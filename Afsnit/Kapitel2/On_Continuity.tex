\subsection{Continuity of Finite Elements}
When looking for a solution a problem, 
 different requirements on the solution 
will impact how the finite elments used to solve an 
variational problem are constructed.

If we require the solution to be continuous, each part of the 
solution must "match up", coming from each element. 
See the following example.

\begin{exmp}{\quad\label{exmp:c0_triangle_con}}
For some $\OO$, let $\mathcal{T}$ be an admissible partition into 
triangles. 
Let $t >0$, and for each finite element, let $\Pi = \mathcal{P}_t$.
In each $T_i$ place
$s = (t+1)(t+2)/2$ points such that:
\begin{itemize}
    \item The points create a grid
    \item There are $t+1$ points on each edge of $T_i$
    \item Every point not on an edge is on an orthogonal intersection of lines drawn between points on the edges 
\end{itemize} %Figur her?
By choosing values for each point, every polynomial on each $T_i$ 
is uniquely determined (the proof of uniqueness can be found in 
\cite{Braess}, page 64).
By restricting the polynomials to an edge, they become polynomials of 
one variable, and are uniquely determined by the values at the $t+1$ points 
on that edge.

Since the values and the points at each edge is the same, polynomials from 
neighbouring elements reduce to the same polynomial, and we get global 
continuity.
\end{exmp}
Example~\ref{exmp:c0_triangle_con} demonstrates, that global continuity are 
somewhat easy to obtain, and constructing finite elements using polynomials 
of any degree grants this property. Letting $t=1$ results in finite elements 
where $\dim(\Pi_i)=3$, and finding the solution becomes computationally 
easy.



If, however, we search for solutions which 
are $C^1(\OO)$, difficulty increases. Along each edge the functions 
themselves must match up, but the first derivatives must now also match. 
We express this using the normal derivative. 
See the following for an example.
\begin{exmp}{\quad\label{exmp:c1_triangle_con}}
   As in Example \ref{exmp:c0_triangle_con}, 
    for some $\OO$, let $\mathcal{T}$ be an admissible partition into 
    triangles, and let $\Pi_i = \mathcal{P}_5$ for each finite element.
    Remember that $\dim(\mathcal{P}_5) = 21$.
    Let the values of the derivatives at each vertice be given up 
    to the $2$nd order, as well as the value of the normal derivative at the 
    midpoint of each element.
    
    To ensure a solution in $C^1(\OO)$, 
    we now check three things: The solution inside each element, 
    on the vertices, and on the edge.

    Inside each element, the solution is $5$th degree polynomial, 
    and therefore also $C^1(T_i)$.

    At each vertice there is $6$ derivatives up to order $2$, which 
    determines a $5$th degree polynomial uniquely, and grants first 
    degree differentiablity.

    Along the edge, the polynomials reduce to $1$ variable, and the 
    normal derivative is a $4$th degree polynomial. Since it follows 
    the derivatives up to the first order at each end of the edge, and 
    the value is given at the midpoint, the normal derivative is uniquely 
    determined as well.

    Thus the normal derivative is continuous everywhere, resulting in a 
    solution in $C^1(\OO)$. This element is well known, and called 
    Argyris element.
\end{exmp}
As we can see from Example \ref{exmp:c1_triangle_con}, by requiring a 
solution be differentiable, construction of finite elements and computation 
becomes harder. 

Many different types of finite elements exists, and can differ wildly in how they 
are constructed. In Example \ref{exmp:c0_triangle_con} and \ref{exmp:c1_triangle_con} 
the finite elements are constructed using a nodal system - specifying all elements of $\Pi$ 
uniquely through values at a number of points. However something like Hsieh–Clough–Tocher element 
is constructed by subdividing triangles further.

%SOMETHING SOMETHING PARALELLOGRAM BAD HERE S. 68