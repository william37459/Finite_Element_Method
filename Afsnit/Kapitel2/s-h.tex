The idea of $S_h$ is to solve the same variational problem as in the global case, 
but in a smaller space - meaning solving
\begin{equation*}
    J(v) = \frac{1}{2}a(v,v) - \ell(v) \to \underset{S_h}{\min}.
\end{equation*}
The only difference is in the space we look for a solution, namely $S_h$. From 
earlier theory, we know that the solution is given by 
\begin{equation*}
    a(u_h,v) = \ell(v) \quad \forall v \in S_h.
\end{equation*}
Since we assumed $S_h$ to be finite, we can let $ \{ \psi_1, \psi_2, \ldots, \psi_N \}$ 
be a basis for $S_h$. Since $\ell$ is a linear functional, 
the solution is 
\begin{equation*}
    a(u_h, \psi_i) = \ell(\psi_i) \quad i = 1, 2, \ldots, N.
\end{equation*}
As $u_h \in S_h$, $u_h$ can be found to be a linear combination of the basis, with 
coefficients $z_i$, $i=1, 2, \ldots, N$. As $a$ is bilinear, we get the system of 
equations
\begin{equation*}
    \sum_{k=1}^N z_k a(\psi_k,\psi_i) = \ell(\psi_i) \quad i = 1,2,\ldots,N.
\end{equation*}
Letting $a(\psi_k,\psi_i)$ be the $i,j$-entries in a matrix $A$, $z_i$ and $\ell(\psi_i)$ 
be the entries the vectors $z$ and $b$ respectively, this can be written in the matrix form 
\begin{equation*}
    Az = b.
\end{equation*}