\begin{thmx}{Aubin–Nitsche \label{thm:aubin-nitsche}}
    Let $H$ be a Hilbert space with norm $|\cdot|$ and a scalar product $(\cdot,\cdot)$.
    Let $V\subset H$ be a Hilbert space for another norm $\|\cdot\|$, and let 
    $V \hookrightarrow H$, the imbedding from $V$ to $H$ be continuous,
    and $\forall g \in H$, let $\varphi_g \in V$ denote the unique weak solution to 
    \begin{equation}
        a(w,\varphi_g) = (g,w) \quad \forall w\in V. \label{eq:aubin_nitsche_antagelse}
    \end{equation}
    Here $a(\cdot,\cdot)$ is a bilinear continuous form.
    Then the finite element solution $u_h\in S_h \subset V$ obeys
    \begin{equation*}
        |u-u_h| \leq C \|u-u_h\| \sup_{g\in H} \left \{ \frac{1}{|g|} \inf_{v \in S_h} \|\varphi_g - v\| \right \},
    \end{equation*}
    where $\sup$ is over all $g\in H$ such that $|g|\neq 0$.
\end{thmx}
\begin{bev}
    Assume $|g|\neq 0$ for some $g\in H$. Then,
    by Cauchy–Schwarz, we can write 
    \begin{equation*}
        (g,w) \leq |g| \cdot |w| \implies \frac{(g,w)}{|g|} \leq |w|.
    \end{equation*}
    By completeness of $H$, if take $\sup$ over all $g\in H$ such that 
    $|g|\neq 0$ (as in the theorem), we still get a $g\in H$, and get 
    an equality, meaning we can write:
    \begin{equation}
        \sup_{g\in H} \frac{(g,w)}{|g|} = |w|. \label{eq:duality_argument}
    \end{equation}
    The solution $u$ and finite element solution $u_h$ is given by 
    \begin{align*}
        a(u,v) &= f(v) \quad \forall v \in V, \\
        a(u_h,v) &= f(v) \quad \forall v \in S_h.
    \end{align*}
    From this we get $a(u-u_h,v)=0$, $\forall v\in S_h$, which we use with the assumption in 
    Equation~\ref{eq:aubin_nitsche_antagelse}, and the continuity of $a(\cdot,\cdot)$ to get
    \begin{align*}
        (g,u-u_h) &= a(u-u_h,\varphi _g) \\
                  &= a(u-u_h,\varphi _g) -0 \\
                  &=a(u-u_h,\varphi _g) -a(u-u_h,v) \\
                  &=a(u-u_h, \varphi_g -g) \\
                  &\leq C \|u-u_h\| \cdot \|\varphi_g-v\|.
    \end{align*}
    Using $v\in S_h$ which gives the smallest norm, we get 
    \begin{equation*}
        (g,u-u_h) \leq C \|u-u_h\| \cdot \inf_{v\in S_h} \|\varphi_g-v\|.
    \end{equation*}
    Using Equation~\ref{eq:duality_argument} on $u-u_h$, we then get 
    \begin{align*}
        |u-u_h| &= \sup_{g\in H} \frac{(g,u-u_h)}{|g|} \\
        &\leq C \| u-u_h\| \sup_{g\in H} \left \{ \frac{1}{|g|} \inf_{v \in S_h} \|\varphi_g - v\| \right \}.
    \end{align*}
\end{bev}