When using a transformation from one shape-regular grid to another, we do not get extra terms which can be seen by looking at the geometric interpretation.
Let $F:T_1 \to T_2 : \hat{x} \mapsto B\hat{x} + x_0$ be a bijective affine mapping.\ we define $\rho_1$ and $r_i$ to be the radius of the largest inscribed circle and the radius of the smallest circle containing $T_1$ respectively.
Given $x \in \mathbb{R}^2$, with $||x|| \leq 2 \rho_1$, such that $x = y_1 - z_1$, where $y_1$ and $z_1$ are the points in $T_1$. Since $F(y_1), F(z_1) \in T_2$, we have that $||Bx|| \leq 2r_2$, thus
\begin{equation}\label{eq:6.9}
    ||B|| \leq \frac{r_2}{\rho_1}
\end{equation}

\begin{thmx}{\quad\label{thm:6.4}}
    Let $t \geq 2$, and suppose $\mathcal{T}_h$ is a shape-regular triangulation of $\OO$. Then there exists a constant $c$ such that
    \begin{equation}
        \norm{u - I_h u}_{m,h} \leq c h^{t-m} \norm{u}_{t, H^t(\OO)}, \quad \forall u \in H^t(\OO), \quad 0 \leq m \leq t,
        \label{eq:6.5}
    \end{equation}
    where $I_h u$ denotes the interpolation by a piecewise polynomial of degreee $t-1$.
\end{thmx}

\begin{bev}
    For this proof, we use the $r_i$ as the radius of the smallest circle containing $T_i\in \mathcal{T}_h$, 
    and $\rho_i$ as the radius of the largest inscribed circle in $T_i$.
    We start by proving the following inequality;
    \begin{equation}
        ||u-I_h u||_{m,T_j} \leq c h^{t-m} |u|_{t,T_j}, \quad \forall u \in H^t(T_j)
    \end{equation}
    for every triangle $T_j$ of a shape-regular triangulation $\mathcal{T}_h$. 
    By choosing a reference triangle with $\hat{r} = 2^{-1/2}$, $\hat{\rho} = {(2+\sqrt{2})}^{-1}$ and letting $F : \hat{T} \to T$ with $T = T_j \in \mathcal{T}_h$, 
    we can apply Lemma~\ref{lem:6.2} on the reference triangle and using the transformations formula, Theorem~\ref{thm:transformation}, in both directions, we get
    \begin{align}
        \begin{split}
        ||u-I_h u||_{m,T_j} &\leq c ||B^{-1}||^m |\det B |^{1/2} |\hat{u} - I_h \hat{u}|_{m,T_{\text{ref}}} \\
                            &\leq c ||B^{-1}||^m |\det B |^{1/2} c | \hat{u} |_{t,T_{\text{ref}}} \\
                            &\leq c ||B^{-1}||^m |\det B |^{1/2} \cdot c ||B||^t \cdot |\det B |^{1/2} |u|_{t,T}  \\
                         &\leq c \left( ||B||  \cdot ||B^{-1}||^m \right) ||B||^{t-m} |u|_{t,T}.
        \end{split}
    \end{align}
    Furthermore, by shape regularity, $r_i / \rho_i \leq \kappa$ for all $T_i$ in $\mathcal{T}$, and $||B|| \cdot ||B^{-1}|| \leq \left( 2 + \sqrt{2} \right)\kappa$. 
    From Equation~\eqref{eq:6.9} we get $||B|| \leq h/\hat{\rho}\leq 4h$.
    With all this we have
    \begin{equation}
        ||u-I_h u||_{\ell,T_j} \leq c h^{t-\ell} |u|_{t,T}
    \end{equation}
    Lastly by squaring and summing the terms from $\ell$ to $m$ we get the desired result.
\end{bev}
