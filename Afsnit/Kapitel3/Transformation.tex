\begin{thmx}{Transformation Formula\label{thm:transformation}}
    Let $\OO$ and $\tilde{\OO}$ be affine equivalents; meaning 
    there exists some square, nonsingular matrix $B$, such that 
    \begin{align*}
        F:&\tilde{\OO}\to \OO, \\
        & F(\tilde{x}) = x_0 + B\tilde{x},
    \end{align*}
    which is affine and bijective. For $v\in H^m(\OO)$ define 
    $\tilde{v}(\tilde{x}) = v(F(\tilde{x}))$.
    Then there exists some $c$ such that 
    \begin{equation*}
        |\tilde{v}|_{m,\tilde{\OO}} \leq 
        c \frac{\|B\|^m }{\sqrt{|\det(B)|}}|v|_{m,\OO}.
    \end{equation*}
\end{thmx}
Here $\|B\|$ is the operator norm, and in our case can be thought of 
as 
\begin{equation*}
    \|B\| = \sup\{ \| By\| :\|y\| = 1 \land y \in \tilde{\OO} \}.
\end{equation*}
If $B$ had another domain than $\tilde{\OO}$, that domain would simply 
replace $\tilde{\OO}$ in the definition.
\begin{bev}
   Writing the $m$th derivative as a multilinear form, we get 
   \begin{equation*}
    (D^m\tilde{v}(\tilde{x}))(\tilde{y}_1, \tilde{y}_2, \ldots, \tilde{y}_m )=
    (D^m v(x))(B\tilde{y}_1, B\tilde{y}_2, \ldots, B\tilde{y}_m ).
   \end{equation*} 
We can extract $B$, and get $\|D^m \tilde{v}\|_{\RR^{nm}} \leq \|B\|^m \|D^m v\|_{\RR^{nm}}$.
As 
\[\partial_{i_{1}}\partial_{i_{2}}\ldots \partial_{i_{m}} v = D^m v(e_{i_1},e_{i_2}, \ldots, e_{i_m})\]
is a $nm$ dimensional vector, we get the following;
\begin{align*}
    \sum_{|\alpha|} | \partial ^\alpha \tilde{v} |^2 & \leq n^m \max_{|\alpha|=m} |\partial ^\alpha \tilde{v}|^2 \\
    &\leq n^m \|D^m\tilde{v}\|^2 \\
    &\leq n^m \|B\|^{2m}\|D^m v\|^2 \\
    &\leq n^{2m}\|B\|^{2m} \sum_{|\alpha|=m}|\partial ^\alpha v|^2.
\end{align*}
We now wish to integrate; however, under affine linear transformation in integrals, 
we must include the transformation constant, which results in 
\begin{equation*}
    \int_{\tilde{\OO}} \sum_{|\alpha|} | \partial ^\alpha \tilde{v} |^2 d\tilde{x}
    \leq n^{2m}\|B\|^{2m}\int_{\OO} \sum_{|\alpha|=m}|\partial ^\alpha v|^2 |\det B^{-1}| dx.
\end{equation*}
Which, after taking the square root, gives the theorem.
\end{bev}