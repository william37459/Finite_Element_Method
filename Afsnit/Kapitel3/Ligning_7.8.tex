\begin{thmx}{\quad\label{thm:max_error}}
    Let the $L_\infty$-norm be 
    \begin{equation*}
        \|v\|_{\infty, \OO} = \esssup_{x\in\OO} |v(x)|.
    \end{equation*}
    Then 
    for a $H^2$-regular variational problem with a solution $u$, the following holds;
    \begin{equation}
        \|u-u_h\|_{\infty, \OO}\leq ch|u|_2.
    \end{equation}
\end{thmx}
\begin{bev}
    For a function $v\in H^2(T_{\text{ref}})$, we let $Iv$ be its interpolant in the polynomial space $\Pi_{\text{ref}}$.
    We have that $H^2\subset C^0$ and therefore 
    \begin{equation}
        \|v-Iv\|_{\infty,T_\text{ref}} \leq c|v|_{2,T_\text{ref}}
        \label{eq:7.9}
    \end{equation}
    by Lemma~\ref{lem:Bramble-Hilbert}.
    Now let $u$ be the solution to the variational problem, and $I_h u$ be the interpolant in $S_h$.
    Then we choose an element $T$ in the triangulation, which we without loss of generality assume to be uniform. 
    Let $\hat{u}$ be the affine transformation of $u|_T$ to the reference triangle.
    Then by Equation~\ref{eq:7.9} and the transformation formula,~\ref{thm:transformation};
    \begin{align}
        \begin{split}
            \|u-I_h u\|_{\infty,T} &= \|\hat{u} - I\hat{u}\|_{\infty,T_{\text{ref}}} \\
            &\leq c|\hat{u}|_{2,T_{\text{ref}}} \\
            &\leq ch|u|_{2,T} \\
            &\leq ch|u|_{2,\Omega}.
        \end{split}
    \end{align}
    From taking the maximum over all triangles, we get
    \begin{equation}
        \|u - I_h u \|_{\infty,\Omega} \leq ch|u|_{2,\Omega}.
        \label{eq:7.10}
    \end{equation}
    Then by the affine argument, we can obtain the inverse estimate as
    \begin{equation}
        \|v_h\|_{\infty,\Omega} \leq ch^{-1} \|v_h\|_{0,\Omega} \quad \text{for all } v_h\in S_h.
    \end{equation}
    Then using Theorem~\ref{thm:6.4} and Corollary~\ref{kor:7.7}
    for $u_h-I_h u\in S_h$
    we get
    \begin{align*}
        \|u_h-I_h u\|_{0,\Omega} &= \| (u-I_h u)-(u-u_h)\|_{0,\Omega} \\
        & \leq \| u-I_h u\|_{0,\Omega}+\|u-u_h\|_{0,\Omega} \\
        &\leq ch^2|u|_{2,\Omega}.
    \end{align*}
    We then use the inverse estimate from earlier. Thus 
    \begin{align}
        \|u-u_h\|_{\infty,\Omega} &\leq \|u-I_h u\|_{\infty,\Omega} + \|u_h-I_h u\|_{\infty,\Omega} \\
        &\leq \|u-I_h u\|_{\infty,\Omega} + ch^{-1} \|u_h-I_h u\|_{0,\Omega} \\
        &\leq ch|u|_{2,\Omega} + ch^{-1} ch^2|u|_{2,\Omega}\\ 
        &\leq \tilde{c}h|u|_{2,\Omega},
    \end{align}
    which is the theorem.    
\end{bev}
Theorem~\ref{thm:max_error} gives us, as the observant reader might have guessed, 
a maximum size of the error of our FEM approximation.
If the variational problem is sufficiently regular (meaning $H^2$)
the error should
become smaller as $h$ becomes smaller. 
We have in this chapter focused on the possible error of our FEM approximation, 
and how big it can be. 
When the variational problem is regular, we have found both an upper maximum 
in general for the error, but we have also shown other error estimates, 
if the problem is not that easy to work with.
We now proceed to the next chapter, and examine an implimentation of 
FEM.