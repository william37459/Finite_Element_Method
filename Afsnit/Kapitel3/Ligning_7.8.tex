We will now prove the statement
\begin{equation}
    \|u-u_h\|_{\infty}\leq ch|u|_2.
\end{equation}
\begin{bev}
    For a function $v\in H^2(T_{\text{ref}})$, we let $Iv$ be its interpolant in the polynomial space $\Pi_{\text{ref}}$.
    We have that $H^2\subset C^0$ and therefore 
    \begin{equation}
        \|v-Iv\|_{\infty,T_\text{ref}} \leq c|v|_{2,T_\text{ref}}
        \label{eq:7.9}
    \end{equation}
    from the Bramble-Hilbert lemma \ref{lem:Bramble-Hilbert}.
    Now let $u$ be the solution to the variational problem, and $I_hu$ be the interpolant in $S_h$.
    Then we choose an element $T$ in the triangulation, which we without loss of generality assume to be uniform. 
    Let $\hat{u}$ be the affine transformation of $u|_T$ to the reference triangle.
    Then by Equation \ref{eq:7.9} and the transformation formula %FIXME label her
    \begin{align}
        \begin{split}
            \|u-I_hu\|_{\infty,T} &= \|\hat{u} - I\hat{u}\|_{\infty,T_{\text{ref}}} \\
            &\leq c|\hat{u}|_{2,T_{\text{ref}}} \\
            &\leq ch|u|_{2,T} \\
            &\leq ch|u|_{2,\Omega}.
        \end{split}
        \label{eq:7.10}
    \end{align}
    From taking the maximum over all triangles, we get
    \begin{equation}
        \|u - I_hu \|_{\infty,\Omega} \leq ch|u|_{2,\Omega}.
    \end{equation}
    Then by the affine argument, we can obtain the inverse estimate as
    \begin{equation}
        \|v_h\|_{\infty,\Omega} \leq ch^{-1} \|v_h\|_{0,\Omega} \quad \text{for all } v_h\in S_h.
    \end{equation}
    Then using Theorem \ref{thm:6.4} and Corollary %FIXME label her
    we get
    \begin{equation}
        \|u-I_hu\|_{0,\Omega} \leq ch^2|u|_{2,\Omega}\quad \text{for } u_h-I_hu\in S_h.
    \end{equation}
    We then use the inverse estimate from earlier. Thus 
    \begin{align}
        \|u-u_h\|_{\infty,\Omega} &\leq \|u-I_hu\|_{\infty,\Omega} + \|u_h-I_hu\|_{\infty,\Omega} \\
        &\leq \|u-I_hu\|_{\infty,\Omega} + ch^{-1} \|u_h-I_hu\|_{0,\Omega}.
    \end{align}
    The equations \ref{eq:6.5} and \ref{eq:7.10} then proves the statement.
\end{bev}