We use the previous theorem to prove an inequality more relevant to our situation.
\begin{kor}{\quad\label{kor:7.7}}
   Assume $\mathcal{T}_h$ is a family of shape regular triangulations of $\OO$.
   If $u\in H^1(\OO)$ is the solution of the variational problem, then 
   \begin{equation*}
    \|u-u_h\|_0 \leq cCh \|u-u_h\|_1.
   \end{equation*}
   If $f\in L_2(\OO)$ and $u\in H^2(\OO)$, then 
   \begin{equation*}
    \|u-u_h\|_0 \leq cC^2h^2 \|f\|_0.
   \end{equation*}
\end{kor}
\begin{bev}
   We want to use Theorem~\ref{thm:aubin-nitsche}, which we can by 
   the following observations. Set 
    \begin{equation*}
        H = H^0(\OO) \quad \text{and} \quad V = H^1_0(\OO).
    \end{equation*}
    Then $V \subset H$, and the norms 
    \begin{equation*}
        |\cdot| = \|\cdot\|_0 \quad \text{and} \quad \|\cdot\| = \|\cdot\|_1,
    \end{equation*}
    by $\|\cdot\|_0 \leq \|\cdot\|_1$ gives us continuity of the imbedding.
    We then get
    \begin{equation*}
        \|u-u_h\|_0 \leq C \|u-u_h\|_1 \sup_{g\in H} \left \{ \frac{1}{|g|} \inf_{v \in S_h} \|\varphi_g - v\|_1 \right \},
    \end{equation*}
    By shape regularity, we can use Theorem~\ref{thm:7.3} and evaluate 
    \begin{equation*}
        \sup_{g\in H} \left \{ \frac{1}{|g|} \inf_{v \in S_h} \|\varphi_g - v\|_1 \right \}
        \leq ch,
    \end{equation*}
    and Theorem~\ref{thm:aubin-nitsche} implies the result.
\end{bev}
If the variational problem is sufficiently regular,
Corollary~\ref{kor:7.7} gives us a connection between $f$ in our original problem, 
and how big the error of our FEM approximation can be. 
However, the norms used to measure this error does not describe the error %! .
at singular points; there still might be points where the error grows uncontrollably large, 
if we just use Corollary~\ref{kor:7.7}. 
To remedy that we use the next theorem. 