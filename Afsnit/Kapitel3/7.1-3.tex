If a domain is regular enough it results in more requirements for the solution, which 
we will introduce here.
To introduce the Regularity Theorem, we must first define when a domain can be classified as regular

\begin{defn}{\quad\label{defn:7.1}}
    Let $m\geq 1, H^m_0(\Omega)\subset V\subset H^m(\Omega)$, and suppose $a(\cdot,\cdot)$ is a $V$-elliptic bilinear form. 
    Then the variational problem
    \begin{equation}
        a(u,v) = {(f,v)}_0\quad \text{for all } v\in V
    \end{equation}
    is called $H^s$-regular provided that there exists a constant $C$ such that for every  $f\in H^{s-2m}(\Omega)$,
    there is a solution $u\in H^s(\Omega)$ satisfying
    \begin{equation}
        \|u\|_s\leq c\|f\|_{s-2m}.
    \end{equation}
\end{defn}
We now introduce the theorem which uses regularity, without proof. Further information can be found in \cite{Braess}.
\begin{thmx}{Regularity Theorem}
    Let $a(\cdot,\cdot)$ be a $H^1_0$-elliptic bilinear form with sufficiently smooth coefficient functions. Then the following holds; 
    \begin{enumerate}
        \item if $\Omega$ is convex the Dirichlet problem is $H^2$-regular, 
        \item if $\Omega$ has a $C^s$ boundary with $s\geq 2$ the Dirichlet problem is $H^s$-regular.
    \end{enumerate}
\end{thmx}

%\textbf{ANTAG HER, AT $\OO$ ER POLYGONELT OG KONVEKS}

From this point on we will assume $\OO$ to be convex and polygonal to ensure that a triangulation of $\OO$ is possible. 
In extension to this we can now look at the error of the finite element approximation.
To do this, let $\mathcal{T}_h$ be a triangulation of $\OO$, and
\begin{align*}
    \mathcal{M}^k_0 =& \{ v \in L_2(\OO) \,|\, v|_T \in \mathcal{P}_k \text{ for every } T \in \mathcal{T}_h \} \cap C^0(\OO) \\
    =& \{ v \in L_2(\OO) \,|\, v|_T \in \mathcal{P}_k \text{ for every } T \in \mathcal{T}_h \} \cap H^1(\OO) .
\end{align*}

\begin{thmx}{\quad}
    Suppose $\mathcal{T}_h$ is a family of shape-regular triangulation of $\Omega$. 
    Then the finite element approximation $u_h\in S_h = \mathcal{M}^k_0$, $k\geq 1$ satisfies
    \begin{align}
        \begin{split}
            \|u-u_h\|_1&\leq ch\|u\|_2\ \\
            &\leq ch\|f\|_0.
        \end{split}
        \label{eq:7.3}
    \end{align}\label{thm:7.3}
\end{thmx}
\begin{bev}
    Since $\Omega$ is convex the problem is $H^2$-regular and that $\|u\|_2 \leq c_1\|f\|_0$.
    From Theorem~\ref{thm:6.4} some $v_h\in S_h$ exists, which satisfies
    \begin{equation}
    \|u-v_h\|_{1,\Omega} = \|u-v_h\|_{1,h} \leq c_2h\|u\|_{2,\Omega}.
    \end{equation}
    Combining the prior statements with Céa's Lemma~\ref{lem:cea}, and 
    define $c = (1+c_1)c_2C/\alpha$ 
    to obtain Equation~\ref{eq:7.3}.
\end{bev}