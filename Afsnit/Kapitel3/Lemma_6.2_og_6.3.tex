
\begin{lem}{\quad~\label{lem:6.2}}
Let $\Omega$ be a domain with a Lipschitz continous boundary which satisfies a cone condition. Furthermore, let $t\geq 2$, and suppose $z_1,z_2,\ldots z_s$ are the
$s = t(t+1)/2$ prescribed points in $\bar{\Omega}$ such that the interpolation operator $I:H^t \rightarrow \mathcal{P}_{t-1}$ is well defined.
Then there exists a constant $c$ such that
\begin{equation}
    \|u-Iu\|_t\leq c\|u\|_t\quad \forall u\in H^t(\Omega).
    \label{eq:lem_6.2}
\end{equation}
\end{lem}

\begin{bev}
    We endow $H^t(\Omega)$ with the norm 
    \begin{equation*}
        |||v||| = |v|_t + \sum_{i=1}^s |v(z_i)|.
    \end{equation*}
It can then be shown that the norms $|||\cdot|||$ and $\|\cdot\|_t$ are equivalent on $H^t(\Omega)$. Thus
\begin{align*}
    \|u-Iu\|_t&\leq c|||u-Iu|||\\
    &= c(|u-Iu|_t + \sum_{i=1}^s|(u-Iu)(z_i)|)\\
    &= c|u-Iu|_t \\
    &= c|u|_t.
\end{align*}
This holds due to the fact that $Iu$ is a polynomial of degree $t-1$, which means that $D^{\alpha} Iu = 0 \text{ forall } |\alpha|=0$, thus $(Iu)(z_1)=u(z_i)$.

We now move on to proving one direction of the equivalence of the norms. 
It can be shown that the imbedding $H^t\hookrightarrow H^2 \hookrightarrow C^0$ is continous (see~\cite{Braess} page 49). Thus
\begin{equation*}
    |v(z_i)|\leq c\|v\|_t\quad \text{for } i=1,2,\ldots,s.
\end{equation*} 
and
\begin{equation*}
    |||v|||\leq (1+cs)\|v\|_t.
\end{equation*}
For the opposite direction, suppose 
\begin{equation*}
    \|v\|_t\leq c|||v|||\quad \text{for all } v\in H^t(\Omega)
\end{equation*}
fails for every positive number $c$. Then there exists some sequence $(v_k)$ in $H^t(\Omega)$ such that
\begin{equation*}
    \|v_k\|_t=1,\quad |||v_k|||\leq \frac{1}{k},\quad k=1,2,\ldots
\end{equation*}
Now it can be shown that a subsequence of $(v_k)$ converges in $\in H^{t-1}(\Omega)$ (see~\cite{Braess} page 32). We can therefore assume that $(v_k)$ itself converges, thus $(v_k)$ is a Cauchy sequence in $H^{t-1}(\Omega)$.
This together with the fact that $|v_k|_t\to 0$ and
 %$\|v_k-v_\ell\|_t^2\leq\|v_k-v_\ell\|_{t-1}^2 + (|v_k|_t + |v_l|_t)^2$, 
\begin{align*}
    \|v_k-v_\ell\|_t^2 &= \|v_k-v_\ell\|_{t-1}^2 + |v_k-v_\ell|_t^2 \\
    & \leq\|v_k-v_\ell\|_{t-1}^2 + {(|v_k|_t + |v_\ell|_t)}^2 ,
\end{align*}
 we get that $(v_k)$ is a Cauchy sequence in $H^t(\Omega)$, and thus converges in $H^t(\Omega)$.
Thus by the continuity, we get 
\begin{equation}
    \|v^*\|_t = 1\quad \text{and }|||v^*|||=0. 
\end{equation}
If $|v^*|_t = 0$ this implies that $v^*$ is a polinomial of degree $t-1$, and from $v^*(z_i)=0$ for $i=1,2,\ldots,s$, we get $v^*=0$. 
This is a contradiction, which proves the equivalence of the norms. %TODO Contradiction af hvad??
\end{bev}

Using Lemma~\ref{lem:6.2}, we can now define and prove the Bramble-Hilbert Lemma.

\begin{lem}{Bramble-Hilbert Lemma}
    Let $\Omega$ be a domain with a Lipschitz continous boundary. Suppose $t\geq 2$ and that $L$ is a bounded linear mapping from $H^t(\Omega)$ into a normed linear space $Y$.
    If $\mathcal{P}_{t-1}\subset \ker L$, then there exists a constant $c$ such that~\label{lem:Bramble-Hilbert}
    \begin{equation}
        \|Lv\|\leq c|v|_t\quad \forall v\in H^t(\Omega).
    \end{equation}
\end{lem}
\begin{bev}
    Let $I:H^t(\Omega)\rightarrow \mathcal{P}_{t-1}$ be an interpolation operator of the same type as in Lemma~\ref{lem:6.2}.
     Then by the same lemma and the fact that $Iv\in\ker L$, we get
    \begin{equation}
        \|Lv\|=\|L(v-Iv)\|\leq \|L\|\cdot\|v-Iv\|_t\leq c\|L\| \cdot |v|_t,
    \end{equation}
    where the constant $c$  is the same as in Equation~\ref{eq:lem_6.2}.~\label{lem:6.3}
\end{bev}