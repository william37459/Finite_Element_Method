\setcounter{secnumdepth}{5}
\bibliographystyle{Bibliografi/vancouver}

\setcitestyle{square}

\addto\captionsdanish{
	\renewcommand\contentsname{Table of Contents}	
	\renewcommand{\bibname}{References}
}


%HEADER
\pagestyle{fancy}
\fancyhf{}
\renewcommand{\chaptermark}[1]{ \markboth{\thechapter.\ #1}{}}
\fancyheadoffset{0pt}

\lhead{\nouppercase \leftmark}
\rhead{Aalborg University}

\renewcommand{\chaptermark}[1]%
        {\markboth{#1}{}}
\renewcommand{\sectionmark}[1]%
        {\markright{\thesection\ #1}}

\lfoot[\fancyplain{}{\bfseries\thepage}]%
    {\fancyplain{}{}}
\rfoot[\fancyplain{}{}]%
    {\fancyplain{}{\bfseries\thepage}}


%KAPITEL
\patchcmd{\chapter}{plain}{fancy}{}{}
\definecolor{gray75}{gray}{0.75}
\newcommand{\hsp}{\hspace{15pt}}
\titleformat{\chapter}[hang]{\huge\bfseries}{\thechapter\hsp\textcolor{gray75}{|}\hsp}{0pt}{\huge\bfseries}
\titlespacing*{\chapter}{0pt}{5pt}{25pt}

\newmdtheoremenv{theo}{Sætning}[section]

% %Sætninger, definitioner, mm.
% \declaretheoremstyle[
%     spaceabove=14pt, 
%     spacebelow=6pt, 
%     headfont=\normalfont\bfseries, 
%     bodyfont = \normalfont,
%     postheadspace=2mm, 
%     headpunct={.}]{mystyle}


% %Sætning    
% \declaretheorem[name={Theorem}, style=mystyle,numberwithin=section]{thmx}
% %Definition
% \declaretheorem[name={Definition}, style=mystyle,sibling=thmx]{defn}
% %Eksempel
% \declaretheorem[name={Example}, style=mystyle,sibling=thmx]{exmp}
% %Lemma
% \declaretheorem[name={Lemma}, style=mystyle,sibling=thmx]{lem}
% %Proposition
% \declaretheorem[name={Proposition}, style=mystyle,sibling=thmx]{pro}
% %Korollar
% \declaretheorem[name={Corollary}, style=mystyle,sibling=thmx]{kor}

\usepackage{framed}
\definecolor{myGray}{HTML}{F9F9F9}

%--------

\renewenvironment{leftbar}[4][\hsize]
{
    \def\FrameCommand
    {
        {\color{#2}\vrule width #4pt}
        \hspace{-8pt}
        \fboxsep=\FrameSep\colorbox{#3}
    }
    \MakeFramed{\hsize#1\advance\hsize-\width\FrameRestore}
}
{\endMakeFramed}

\usepackage{array}
\usepackage{makecell}

\renewcommand\theadalign{bc}
\renewcommand\theadfont{\bfseries}
\renewcommand\theadgape{\Gape[4pt]}
\renewcommand\cellgape{\Gape[4pt]}


\theoremstyle{definition}
\newtheorem{thm}{Theorem}[chapter]
\newtheorem{lemm}[thm]{Lemma}
\newtheorem{pro}[thm]{Proposition}
\newtheorem{cor}[thm]{Corollary}
\newtheorem{defi}[thm]{Definition}
\newtheorem{conj}[thm]{Conjecture}
\newtheorem{eks}[thm]{Example}

\usepackage{changepage}



%----

\newenvironment{thmx}[1]
    {\begin{leftbar}{black}{myGray}{3}
    \begin{thm}#1\\
        }{
    \end{thm}
    \end{leftbar}
    }
    
\newenvironment{prop}[1]
    {\begin{leftbar}{black}{myGray}{3}
    \begin{pro}#1\\
        }{
    \end{pro}
    \end{leftbar}
    }
    
\newenvironment{lem}[1]
    {\begin{leftbar}{black}{myGray}{3}
    \begin{lemm}#1\\
        }{
    \end{lemm}
    \end{leftbar}
    }
  
\newenvironment{defn}[1]
    {\begin{leftbar}{black}{myGray}{3}
    \begin{defi}#1\\
        }{
    \end{defi}
    \end{leftbar}
    }
    
    \newenvironment{kor}[1]
    {\begin{leftbar}{black}{myGray}{3}
    \begin{cor}#1\\
        }{
    \end{cor}
    \end{leftbar}
    }
    
\newenvironment{exmp}[1]
    {\begin{leftbar}{gray}{white}{2}
    \begin{eks}#1\\
        }{
    \end{eks}
    \end{leftbar}
    }



%Bevis
\newtheoremstyle{beviss}% name of the style to be used
  {0pt}% measure of space to leave above the theorem. E.g.: 3pt
  {14pt}% measure of space to leave below the theorem. E.g.: 3pt
  {}% name of font to use in the body of the theorem
  {0pt}% measure of space to indent
  {\bfseries}% name of head font
  {.}% punctuation between head and body
  { }% space after theorem head; " " = normal interword space
  {\thmname{#1}}
\theoremstyle{beviss}
\newtheorem{bev}{Proof}
\AtEndEnvironment{bev}{\null\hfill$\blacksquare$}%

\makeatletter
\pretocmd{\chapter}{\addtocontents{toc}{\protect\addvspace{-2\p@}}}{}{}
\pretocmd{\section}{\addtocontents{toc}{\protect\addvspace{0\p@}}}{}{}
\makeatother

\titlespacing*{\subsection}
{0pt}{1em}{1em}




