\documentclass[10pt]{beamer}
\usetheme[
%%% options passed to the outer theme
%    hidetitle,           % hide the (short) title in the sidebar
%    hideauthor,          % hide the (short) author in the sidebar
%    hideinstitute,       % hide the (short) institute in the bottom of the sidebar
%    shownavsym,          % show the navigation symbols
%    width=2cm,           % width of the sidebar (default is 2 cm)
%    hideothersubsections,% hide all subsections but the subsections in the current section
%    hideallsubsections,  % hide all subsections
%    left                % right of left position of sidebar (default is right)
  ]{Aalborg}
  
% If you want to change the colors of the various elements in the theme, edit and uncomment the following lines
% Change the bar and sidebar colors:
%\setbeamercolor{Aalborg}{fg=red!20,bg=red}
%\setbeamercolor{sidebar}{bg=red!20}
% Change the color of the structural elements:
%\setbeamercolor{structure}{fg=red}
% Change the frame title text color:
%\setbeamercolor{frametitle}{fg=blue}
% Change the normal text color background:
%\setbeamercolor{normal text}{bg=gray!10}
% ... and you can of course change a lot more - see the beamer user manual.

\usepackage[utf8]{inputenc}
\usepackage[english]{babel}
\usepackage[T1]{fontenc}
% Or whatever. Note that the encoding and the font should match. If T1
% does not look nice, try deleting the line with the fontenc.
\usepackage{helvet}

% colored hyperlinks
\newcommand{\chref}[2]{%
  \href{#1}{{\usebeamercolor[bg]{Aalborg}#2}}%
}

\title[An introductory examination of
the Finite Element Method]% optional, use only with long paper titles
{An introductory examination of
the Finite Element Method}


\date{10. Juni, 2024}

\author[Jacob Engberg, Patrick Guldberg, William Dam] % optional, use only with lots of authors
{Jacob Engberg,\\ Patrick Guldberg,\\ William Dam
}
% - Give the names in the same order as they appear in the paper.
% - Use the \inst{?} command only if the authors have different
%   affiliation. See the beamer manual for an example

\institute[
%  {\includegraphics[scale=0.2]{aau_segl}}\\ %insert a company, department or university logo
  Institut for Matematiske Fag\\
  Aalborg Universitet\\
  Danmark
] % optional - is placed in the bottom of the sidebar on every slide
{% is placed on the bottom of the title page
  Institut for Matematiske Fag\\
  Aalborg Universitet\\
  Danmark
  
  %there must be an empty line above this line - otherwise some unwanted space is added between the university and the country (I do not know why;( )
}

% specify the logo in the top right/left of the slide
\pgfdeclareimage[height=1cm]{mainlogo}{AAUgraphics/aau_logo_new} % placed in the upper left/right corner
\logo{\pgfuseimage{mainlogo}}

% specify a logo on the titlepage (you can specify additional logos an include them in 
% institute command below
\pgfdeclareimage[height=1.5cm]{titlepagelogo}{AAUgraphics/aau_logo_new} % placed on the title page
%\pgfdeclareimage[height=1.5cm]{titlepagelogo2}{AAUgraphics/aau_logo_new} % placed on the title page
\titlegraphic{% is placed on the bottom of the title page
  \pgfuseimage{titlepagelogo}
%  \hspace{1cm}\pgfuseimage{titlepagelogo2}
}

\begin{document}
% the titlepage
{\aauwavesbg
\begin{frame}[plain,noframenumbering] % the plain option removes the sidebar and header from the title page
  \titlepage
\end{frame}}
%%%%%%%%%%%%%%%%

% TOC
\begin{frame}{Finite Element Method}{General Idea}
    \begin{itemize}
        \item Partition the domain $\Omega$.
        \item Define subspace with finite dimension $S_h$.
        \item Elements/Cells.
    \end{itemize}
\end{frame}

\begin{frame}{Finite Element Method}{Partitioning}
    \begin{itemize}
        \item Splicing functions over each cell.
        \item Edge restrictions.
        \item Solve the varational problem over $S_h$.
        \begin{equation}
            J(v) = \frac{1}{2} a(v,v) - \ell(v) \rightarrow \min_{S_h}.
        \end{equation}
        \item Solution $u_h \in S_h$,
        \begin{equation}
            a(u_h,v) = \ell(v) \quad \forall v \in S_h.
        \end{equation}
    \end{itemize}
\end{frame}

\begin{frame}{Finite Element Method}{Finite Elements}
    \begin{defn}{Finite Element}
        A finite element is a triple $(T, \Pi,\Sigma)$ which has the following properties:
        \label{def:finite_element}
        \begin{enumerate}
            \item $T\subset \RR^d$ is a polyhedron
            \item $\Pi \subset C(T)$ with finite dimension $s$
            \item $\Sigma$ is a set of $s$ linearly independent functionals on $\Pi$. 
            Every $p\in \Pi$ is uniquely defined by the values of the $s$ functionals in $\Sigma$.
        \end{enumerate}
    \end{defn}
\end{frame}

\begin{frame}{Finite Element Method}{Reference Finite Element}
    \begin{itemize}
        \item Why do we need a Reference Finite Element?
        \item \begin{defn}
            Let $(T_{\text{ref}}, \Pi_{\text{ref}},\Sigma_{\text{ref}})$ be a finite element 
        with $T_{\text{ref}}\in \mathcal{T}$ for som admissible partition of 
        $\OO$, and 
        $F$ an affine transformation. 
        Assume that for $T_i\in\mathcal{T}$ the following 
        is true for the corresponding finite element $(T_i, \Pi_i,\Sigma_i)$:
        \begin{itemize}
            \item $F(T_{\text{ref}}) = T_i$
            \item $\{ f\circ F \,|\,  f \in \Pi_i \} =\Pi_{\text{ref}}  $
            \item $\{ s(f\circ F) \,|\, f \in \Pi_i, s \in \Sigma_{\text{ref}} \} = \Sigma_i$
        \end{itemize}
        If the previous equalities are true for all $T_i\in \mathcal{T}$ we call 
        $(T_{\text{ref}}, \Pi_{\text{ref}},\Sigma_{\text{ref}})$ the finite reference 
        element.
        \end{defn}
    \end{itemize}
\end{frame}

\begin{frame}{Finite Element Method}{Partition into triangles and quadrilaterals}
    \begin{itemize}
        \item Triangulation.
        \item Quadrilateral partition.
    \end{itemize}
    
\end{frame}
\begin{frame}{Agenda}{}
\tableofcontents
\end{frame}
%%%%%%%%%%%%%%%%

\section{PDE og grænsebetingelser}
\subsection{Basis om PDE}
\begin{frame}{Basis om PDE}{}
    \begin{equation*}
        f(x) = c(x) u + \sum_{i=1}^{n}b_i(x)u_{x_{i}}
       - \sum_{i,k=1}^{n}a_{ik}(x)u_{x_i x_k}
   \end{equation*}    

   Elliptisk i $x$, hvis $A(x)$ er positiv definit
\end{frame}

\begin{frame}{Sobolev rum}{}
    Lad $m \geq 0$ være et heltal, så er $H^m(\Omega)$
    \begin{equation*}
        H^m(\Omega) = \{  f \in L_2(\Omega) \mid \partial ^{\alpha}f \in 
        L_2(\Omega) \quad \forall |\alpha| \leq m  \}.
    \end{equation*}
\end{frame}

\subsection{Dirichlet}
\begin{frame}{Homogene Dirichlet Grænsebetingelser}{}
    \begin{align*}
        Lu &= f \quad \text{in } \Omega\\
        u &= 0 \quad \text{on } \partial \Omega.
    \end{align*}
\end{frame}

\begin{frame}{Minimal egenskab}{}
    Lad en elliptisk PDE være givet, og lad $a_{ik}$ være indgangene i den positivt definite matrix $A$ for PDE'en.
    Alle klassiske løsninger af grænseværdi betingelsen givet ved
    \begin{align}
        -\sum_{i,k} \partial_i (a_{ik}\partial_k u) + a_0 u &= f \quad \text{in } \Omega  \\
        u &= 0 \quad \text{on } \partial \Omega,
    \end{align}
        er en løsning til variational problemet givet ved
        \[
            J(v)=\int_\Omega \left [\frac{1}{2}\sum_{i,k} a_{ik} \partial_i v\partial_k v + \frac{1}{2} a_0 v^2 -fv\right ]dx \longrightarrow \min,
        \]
        blandt alle funktioner i $C^2(\Omega)\cap C^0(\bar{\Omega})$ med nul grænse værdier.
\end{frame}


\begin{frame}{Eksistens sætning}{}
    Lad $L$ være en anden ordens uniformt elliptisk differential operator. Så har det homogene Dirichlet problem altid en svag løsning i $H_0^1(\OO)$. Det er et minimum af variational problemet
   \begin{equation}
       J(v)=\frac{1}{2} a(v,v) - {(f, v)}_0 \rightarrow \text{min}
   \end{equation}
   på $H_0^1(\OO)$.
\end{frame}


\begin{frame}{Eksistens sætning}{Bevis}
    Den uniforme ellipticitet medfører følgende punktvise estimat
    \[
        \sum_{i,k} a_{ik} \partial_i v \partial_k v \geq \alpha \sum_i {\left( \partial_i v \right)}^2,
    \]
    for $C^1(\OO)$ funktioner.\\
    Pr tæthedsargumentet er $C^1(\OO)$ tæt i $H^1(\OO)$, kan vi generelaisere antagelserne til $H^1(\OO)$.
    \begin{equation}
        a(v,v) \geq \alpha \sum_i \int_\OO {\left(\partial_i v\right)}^2 dx = \alpha |v|^2_1, \quad \forall v \in H^1(\OO).
    \end{equation}
    Fra Lax-Milgram findes der en entydig svag løsning til variational problemet.
\end{frame}

\subsection{Neumann}
\begin{frame}{Neumann Grænsebetingelser}{}
    Lad $\frac{\partial u}{\partial \mathbf{n}}= \mathbf{n}\cdot \nabla u$ være den afledede af normalen. Så er Neumann grænsebetingelsen givet ved
    \begin{equation}
        \frac{\partial u}{\partial \mathbf{n}}  = g\quad \text{on } \partial \Omega.
    \end{equation}
\end{frame}

\begin{frame}{Neumann Grænsebetingelser}{Løsning}
    Lad $\OO$ være begrænset, have en stykkevis glat rand, og opfylde cone condition. Lad $f \in L_2(\OO)$ og $g\in L_2(\partial \OO)$. Der eksisterer en entydig $u \in H^1(\OO)$ som løser variational problemet
    \begin{equation*}
     J(v) = \frac{1}{2}a(v,v) - {(f,v)}_{0,\OO} - {(g,v)}_{0,\partial\OO} \to \min.
    \end{equation*}
    Ydermere, $u \in C^2(\OO)\cap C^1(\bar{\OO})$ hvis og kun hvis en klassisk løsning af
    \begin{align}
     Lu &= f \quad \text{ in }\OO, \nonumber \\
     \sum_{i,k}  \mathbf{n}_i a_{ik} \partial_k u &= g \quad \text{ on } \partial \OO, \label{eq:neuman_condition_boundary}
    \end{align}
    eksisterer, og i så fald er de to løsninger de samme. Her er $\mathbf{n}$ den udadgående normal på $\partial \OO$, defineret næsten overalt.
\end{frame}

\section{Finite Element Method}
\subsection{Opdeling}
\begin{frame}{$S_h$}{}
\end{frame}

\subsection{Finite Elements}
\begin{frame}{Konstruktion}{}
\end{frame}
\section{Teoretiske fejl}
\begin{frame}{Inverse Estimater}{}
\end{frame}
\begin{frame}{Aubin–Nitsche og 4.6}{}
\end{frame}

\section{Numerisk Analyse}
\end{document}
