
\section{PDE og grænsebetingelser}
\subsection{Basis om PDE}
\begin{frame}{Basis om PDE}{}
    \begin{equation*}
        f(x) = c(x) u + \sum_{i=1}^{n}b_i(x)u_{x_{i}}
       - \sum_{i,k=1}^{n}a_{ik}(x)u_{x_i x_k}
   \end{equation*}    

   Elliptisk i $x$, hvis $A(x)$ er positiv definit
\end{frame}

\begin{frame}{Sobolev rum}{}
    Lad $m \geq 0$ være et heltal, så er $H^m(\Omega)$
    \begin{equation*}
        H^m(\Omega) = \{  f \in L_2(\Omega) \mid \partial ^{\alpha}f \in 
        L_2(\Omega) \quad \forall |\alpha| \leq m  \}.
    \end{equation*}
\end{frame}

\subsection{Dirichlet}
\begin{frame}{Homogene Dirichlet Grænsebetingelser}{}
    \begin{align*}
        Lu &= f \quad \text{in } \Omega\\
        u &= 0 \quad \text{on } \partial \Omega.
    \end{align*}
\end{frame}

\begin{frame}{Minimal egenskab}{}
    Lad en elliptisk PDE være givet, og lad $a_{ik}$ være indgangene i den positivt definite matrix $A$ for PDE'en.
    Alle klassiske løsninger af grænseværdi betingelsen givet ved
    \begin{align}
        -\sum_{i,k} \partial_i (a_{ik}\partial_k u) + a_0 u &= f \quad \text{in } \Omega  \\
        u &= 0 \quad \text{on } \partial \Omega,
    \end{align}
        er en løsning til variational problemet givet ved
        \[
            J(v)=\int_\Omega \left [\frac{1}{2}\sum_{i,k} a_{ik} \partial_i v\partial_k v + \frac{1}{2} a_0 v^2 -fv\right ]dx \longrightarrow \min,
        \]
        blandt alle funktioner i $C^2(\Omega)\cap C^0(\bar{\Omega})$ med nul grænse værdier.
\end{frame}


\begin{frame}{Eksistens sætning}{}
    Lad $L$ være en anden ordens uniformt elliptisk differential operator. Så har det homogene Dirichlet problem altid en svag løsning i $H_0^1(\OO)$. Det er et minimum af variational problemet
   \begin{equation}
       J(v)=\frac{1}{2} a(v,v) - {(f, v)}_0 \rightarrow \text{min}
   \end{equation}
   på $H_0^1(\OO)$.
\end{frame}


\begin{frame}{Eksistens sætning}{Bevis}
    Den uniforme ellipticitet medfører følgende punktvise estimat
    \[
        \sum_{i,k} a_{ik} \partial_i v \partial_k v \geq \alpha \sum_i {\left( \partial_i v \right)}^2,
    \]
    for $C^1(\OO)$ funktioner.
    Ved at følge de samme tætheds argumenter som i beviset for Friedrichs ulighed, er $C^1(\OO)$ tæt i $H^1(\OO)$, og derfor er alle $u \in H^1(\OO)$ repræsenteret i $C^1(\OO)$, og dermed kan den tidligere antagelse generaliseres til $H^1(\OO)$.
    Ved at integrere begge sider og anvende $\text{a}_0 \geq 0$ får vi
    \begin{equation}
        a(v,v) \geq \alpha \sum_i \int_\OO {\left(\partial_i v\right)}^2 dx = \alpha |v|^2_1, \quad \forall v \in H^1(\OO).
    \end{equation}
    Vi ved fra Friedrichs ulighed at $|\cdot|_1$ og $\| \cdot \|_1$ er ækvivalente normer på $H_0^1$,
    hvilket medfører at $a$ er en $H^1$-elliptisk bilinear form på $H_0^1(\OO)$.
    Fra Lax-Milgram findes der en entydig svag løsning til variational problemet, som også er en løsning til variations problemet.
\end{frame}

\subsection{Neumann}
\begin{frame}{Neumann Grænsebetingelser}{}
    Lad $\frac{\partial u}{\partial \mathbf{n}}= \mathbf{n}\cdot \nabla u$ være den afledede af normalen. Så er Neumann grænsebetingelsen givet ved
    \begin{equation}
        \frac{\partial u}{\partial \mathbf{n}}  = g\quad \text{on } \partial \Omega.
    \end{equation}
\end{frame}

\begin{frame}{Neumann Grænsebetingelser}{}
    Lad $\OO$ være begrænset, have en stykkevis glat rand, og opfylde cone condition. Lad $f \in L_2(\OO)$ og $g\in L_2(\partial \OO)$. Der eksisterer en entydig $u \in H^1(\OO)$ som løser variational problemet
    \begin{equation*}
     J(v) = \frac{1}{2}a(v,v) - {(f,v)}_{0,\OO} - {(g,v)}_{0,\partial\OO} \to \min.
    \end{equation*}
    Ydermere, $u \in C^2(\OO)\cap C^1(\bar{\OO})$ hvis og kun hvis en klassisk løsning af
    \begin{align}
     Lu &= f \quad \text{ in }\OO, \nonumber \\
     \sum_{i,k}  \mathbf{n}_i a_{ik} \partial_k u &= g \quad \text{ on } \partial \OO, \label{eq:neuman_condition_boundary}
    \end{align}
    eksisterer, og i så fald er de to løsninger de samme. Her er $\mathbf{n}$ den udadgående normal på $\partial \OO$, defineret næsten overalt.
\end{frame}