\begin{frame}{Finite Element Method}{General Idea}
    \begin{itemize}
        \item Partition the domain $\Omega$.
        \item Define subspace with finite dimension $S_h$.
        \item Elements/Cells.
    \end{itemize}
\end{frame}

\begin{frame}{Finite Element Method}{Partitioning}
    \begin{itemize}
        \item Splicing functions over each cell.
        \item Edge restrictions.
        \item Solve the varational problem over $S_h$.
        \begin{equation}
            J(v) = \frac{1}{2} a(v,v) - \ell(v) \rightarrow \min_{S_h}.
        \end{equation}
        \item Solution $u_h \in S_h$,
        \begin{equation}
            a(u_h,v) = \ell(v) \quad \forall v \in S_h.
        \end{equation}
    \end{itemize}
\end{frame}

\begin{frame}{Finite Element Method}{Finite Elements}
    \begin{defn}{Finite Element}
        A finite element is a triple $(T, \Pi,\Sigma)$ which has the following properties:
        \label{def:finite_element}
        \begin{enumerate}
            \item $T\subset \RR^d$ is a polyhedron
            \item $\Pi \subset C(T)$ with finite dimension $s$
            \item $\Sigma$ is a set of $s$ linearly independent functionals on $\Pi$. 
            Every $p\in \Pi$ is uniquely defined by the values of the $s$ functionals in $\Sigma$.
        \end{enumerate}
    \end{defn}
\end{frame}

\begin{frame}{Finite Element Method}{Reference Finite Element}
    \begin{itemize}
        \item Why do we need a Reference Finite Element?
        \item \begin{defn}
            Let $(T_{\text{ref}}, \Pi_{\text{ref}},\Sigma_{\text{ref}})$ be a finite element 
        with $T_{\text{ref}}\in \mathcal{T}$ for som admissible partition of 
        $\OO$, and 
        $F$ an affine transformation. 
        Assume that for $T_i\in\mathcal{T}$ the following 
        is true for the corresponding finite element $(T_i, \Pi_i,\Sigma_i)$:
        \begin{itemize}
            \item $F(T_{\text{ref}}) = T_i$
            \item $\{ f\circ F \,|\,  f \in \Pi_i \} =\Pi_{\text{ref}}  $
            \item $\{ s(f\circ F) \,|\, f \in \Pi_i, s \in \Sigma_{\text{ref}} \} = \Sigma_i$
        \end{itemize}
        If the previous equalities are true for all $T_i\in \mathcal{T}$ we call 
        $(T_{\text{ref}}, \Pi_{\text{ref}},\Sigma_{\text{ref}})$ the finite reference 
        element.
        \end{defn}
    \end{itemize}
\end{frame}

\begin{frame}{Finite Element Method}{Partition into triangles and quadrilaterals}
    \begin{itemize}
        \item Triangulation.
        \item Quadrilateral partition.
    \end{itemize}
    
\end{frame}